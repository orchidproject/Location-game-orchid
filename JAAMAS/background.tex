\section{Background}
\todo{Background on
\begin{enumerate}
\item Gopal's background on coordination for disaster response, RoboCup Rescue, and Scerri's work.
\item Feng's background on MDP approaches to coordination (incl CFST and Archie's paper)
\item Joel/Wenchao to add background on HCI studies on coordination in disasters + mixed reality 
\end{enumerate}}

\subsection{ Team Coordination and Disaster Response}
Team coordination can be defined as “the act of managing interdependencies between activities performed to achieve a goal” Malone (1990: 361). In disaster response, team coordination is essential in order that groups of people can carry out interdependent activities together in a timely and satisfactory manner (cf. Bradshaw et al., 2011). Disaster response experts report that “failures in team coordination are the most significant factor in critical emergency response” (Toups et al., 2011: 2) that can cost human lives. Shared understanding, situation awareness, and alignment of cooperative action through on-going communication are key requirements to enable successful coordination. Literatures (Monares, 2011, Padilha, 2010, Convertino et al. 2011) have proposed a set of tools and system architectures that support common ground and awareness in emergency management. \\

(Chen et al., 2005) highlighted that One important characteristic of large-scale disaster is the presence of multiple spatially distributed incidents . To deal with multiple incidents, the disaster response team has to coordinate spatially distributed resources and personnel to carry out operations (e.g. search, rescue and evacuation). Therefore, it is necessary to optimize coordination of teams by allocating tasks to teams in time and space efficiently and sufficiently (Nourjou et al 2011). \\

\subsection{ Issues of Human Agent Collaboration }
Many multi-agent agent coordination algorithms have potential to be applied to support task assignment of responder teams. However, before we use those algorithms to build agent planning support system, there is a need to understand how human and agent support system can effectively collaborate.  Human factors researchers have conducted controlled experiments to identify key aspects of human agent collaboration [ Bradshaw 2011, Cooke2007 ,Sukthankar2009, Wagner2004] and evaluate strategies of agent support for teams [Lenox1998,Lenox2000,(Nourjou et al 2011) ]. Prior research has recognised that interaction design is vital for the performance of socio-technical human-agent systems [Murthy1997], particularly where an agent directly instructs humans [S. Moran2013]. With inappropriate interaction design, agent-based planning support may function inefficiently, or at worst, hinder the performance of human teams. Although there is much literature in planning support, task assignment, and human-agent collaboration, Yet, real world studies of how human teams handle agent support are rare. \\

Moverover, CSCW literatures[Bowers1994] have pointed out ill-designed work-flow management/automation system can lead to undesirable results, not only fail to improve work efficiency but also hinders human performance. Bowers et al. found that extreme difficulties might be encountered when introducing new technology support for human teams. New technologies might not support, but  disrupt smooth workflow if they are designed in an organisationally unacceptable way. [Abbott1994] We believe the same is true for intelligent planning support. Before we can build intelligent systems that support human team coordination, field trials are needed to understand the potential impact of technology support for team coordination. \\

\subsection{ Disaster Simulation and Games }

Computational simulations, particularly agent-based simulations of disasters, are the predominant approach in the computing literature to predict the consequences of cer- tain courses of action (Hawe et al., 2012), to model first responder information flow (Robinson and Brown, 2005), or to model the logistic distribution of emergency relief supplies (Lee et al., 2007).\\

The limitations of the veracity of computational simulations are manifold. For ex- ample, Simonovic highlights that simulations may rely on unrealistic geographical topography, and most importantly, may not account for “human psychosocial charac- teristics and individual movement, and (...) learning ability” (Simonovic, 2009: 89). The impact of emotional and physical responses likely in a disaster, such as stress, fear, exertion or panic (Drury et al., 2009) remains understudied in approaches that rely purely on computational simulation.\\


In this study, we adopt a mixed-reality game approach to put people under realistic cognitive and physical stress. Mixed-reality games are recreational experiences that make use of pervasive technologies such as smart phones, wireless technologies and sensors with the aim of blending game events into a real world environment [6]. Arguably, they have become an established vehicle to explore socio-technical issues in complex real world settings [5]. The major advantage of mixed-reality games is the fact that they are situated in the real world, which arguably leads to increased efficacy of the behavioural observations when compared to computational simulations.\\


\todo{
The following are a few suggestions for improving the work.
-- the paper should cite the work by Humphereys and Adams that discussed assignment of responders in emergency situations. Their work is particularly relevant because of the way they balanced competing task demands.
}

\todo{Please add more references to cover beyond what we do to some extent}