\section{The Disaster Scenario}

\noindent We consider a disaster scenario involving a satellite, containing radioactive fuel, that has crashed in a sub-urban area (see Section \ref{atomic} to see how this helps implement a credible mixed-reality game). While debris is strewn around a large area, damaging buildings and causing accidents and injuring civilians, radioactive discharge from the debris is gradually spreading over the area, threatening to contaminate food reserves and people. Hence, emergency services, voluntary organisations, and the military are deployed to help evacuate the casualties and resources before these are engulfed by  radioactive cloud.  In what follows, we model this scenario formally and then describe the optimisation problem faced by the actors (i.e., including emergency services, volunteers, medics, and soldiers) in trying to save as many lives and resources as they can.

\subsection{Formal Model}
\noindent Let $G$ denote a grid overlaid on top of the disaster space, and the satellite and actors are located at various coordinates $(x,y) \in G$ in this grid. The set of field responders be denoted as $i_1, \cdots, i_n \in I$ and the set of rescue tasks as  $t_1,\cdots, t_m\in T$.  As responders enact tasks, they may become tired or get injured. Hence, we assign each responder  a health level $h_i\in [0,100]$. Moreover, each responder will have  a specific role  $r \in Roles$ (e.g., fire brigade, soldier, or medic) and this will determine the capabilities he or she has and therefore the tasks he or she can perform. We denote as $Roles(i)$ the role of responder $i$. In turn, to complete a given task $t$,  a set of responders $I' \subseteq I$ with specific roles $R_t \subseteq R$ is required. Thus, a task can only be completed by a team of responders $I'$ if $\{Roles(i) | i \in I'\} = R_t$. 

Given this model, we next formulate the optimisation problem faced by the responders (and later solved in Section \ref{sec:algo}). To this end, we propose a Multi-Agent Markov Decision Process (MMDP)~\cite{?} that captures the uncertainties of the radiative cloud and the responders' behaviours. Specifically, we model the spreading of the radiative cloud as a random process over the spatial space and allow the responders' actions to be failed or delayed during the rescue process. Other existing models for rescue tasks such as CFSTP~\cite{?} and HTSSC~\cite{?} cannot be applied to our problem as they usually require the process of task executions to be deterministic. They need to explicitly model the time limit of a task and the duration of completing a task by the responders as spatial and temporal constraints, which are stochastic in our domain. Instead, in the MMDP model, we represent the task executions as stochastic processes of state transitions. Thus, the uncertainties of the radiative cloud and the responders' behaviours can be straightforwardly captured with transition probabilities. Additionally, modeling the problem as a MMDP enables us to use many sophisticated algorithms that have already been developed in the literature.

\subsection{Radiation Cloud Modelling}
The radiation cloud diffusion process is modelled by a nonlinear Markov field stochastic differential equation, which assumes the cloud intensity is Gaussian distributed in log-space.  The cloud is driven by wind forces which vary both spatially and temporally.  Wind forces induce an anisotropic diffusion coefficient into the cloud diffusion process.  The wind velocity is modelled by two a priori independent Gaussian processes (GP), one GP for each Cartesian coordinate axis.  The GP captures both the spatial distribution of the wind velocity and also the dynamic process resulting from shifting wind patterns such as short term gusts and longer term variations.  In our simulation, each spatial wind velocity component is modelled by a squared-exponential GP covariance function, $K$, with fixed input and output scales over time (although any covariance function, stationary or not, can be substituted).

Both the radiation cloud and wind model priors are combined into a single joint model called a {\it latent force model} (LFM)~\cite{alvarez09} and predictions of the radiation cloud intensity are inferred using the extended Kalman filter (EKF).  The EKF provides both the mean and variance of the log-radiation cloud intensity and wind conditions.  Uncertainty arises due to unknown initial conditions of the cloud and wind conditions and it is also induced by the stochastic nature of their processes.  The EKF state $S(t)=(\underline{R}(t) \underline{V}_x(t) \underline{V}_y(t))^T$ represents both Cartesian components of the wind velocity, $V_x(t)$ and $V_y(t)$, and the log-radiation cloud density, $R(t)$, on a regular $N\times M$ grid defined across the environment with grid coordinates $G$.  The temporal component of the wind GP model is assumed Markovian and thus, the wind dynamics are incorporated within the EKF as per the KFGP~\cite{reece10}.  For example, the $N\times M$ x-component of the wind velocity at time-step $t+1$ is $V_x(t+1)=F V_x(t)+\nu_t$, where the process model $F=\rho I$ (where $I$ is the identity matrix) and Gaussian process noise $\nu_t\sim \Bbb{N}(0,(1-\rho^2) K(G,G))$ for correlation, $\rho$, of the wind field between time steps.  When $\rho=1$ the wind velocities are time invariant (although spatially variant).  Values of $\rho<1$ model wind conditions that change over time.

The cloud intensity and wind velocity are measured by {\it monitor agents} equipped with geiger-counters and anemometers.  These agents are directed to take measurements with greatest information gain in the radiation cloud intensity.  The measurements are folded into the EKF and this refines estimates of the radiation cloud across the grid.  Figure~\ref{radiation_screen_shots} shows example cloud simulations for invariant (i.e. $\rho=1.0$) and gusty (i.e. $\rho=0.90$) wind conditions.  Figure~\ref{radiation_screen_shots}(a) shows invariant wind conditions in which case the radiation cloud can be interpolated accurately using sparse sensor measurements and the LFM model.  Alternatively, during gusty conditions the radiation cloud model is more uncertain far from the locations where recent measurements have been taken, as shown in Figure~\ref{radiation_screen_shots}(b).

\begin{figure}[ht] \begin{center}
    \includegraphics[width=0.45\textwidth]{figures/radiation_ss_calm.png}\\
    (a) Slow varying wind conditions\\ \ \\
    \includegraphics[width=0.45\textwidth]{figures/radiation_ss_gust.png}\\
    (b) Gusty wind conditions 
\caption{\label{radiation_screen_shots} Radiation and wind simulation ground truth and EKF estimates obtained using measurements from monitor agents (black dots).  Left most panes are ground truth radiation and wind conditions, the middle panes are corresponding estimates and right most panes are state uncertainties:  (a) Invariant and (b) gusty wind conditions.}
\end{center}
\end{figure}

\subsection{The Optimisation Problem}
\noindent A Multi-agent Markov Decision Process (MMDP) is formally
defined as a tuple $\mathcal{M} = \langle I, S, \{A_i\}, P, R
\rangle$, where:
\begin{itemize}
  \item $I$ is a set of $n$ field responders and each responder
      is associated with a unique identifier number $p_i\in I$.
  \item $S = S_r \times S_{p_1} \times \cdots \times S_{p_n}
      \times S_{t_1} \times \cdots \times S_{t_m}$ is the state
      space. $S_r = \{l_{(x,y)}| (x, y) \in G\}$ is the state
      variable of the radioactive cloud to specify the
      radioactive level $l_{(x,y)}\in[0, 100]$ at every point
      $(x, y)\in G$. $S_{p_i} = \langle h_i, (x_i, y_i), t_j
      \rangle$ is the state variable for each responder $p_i$
      to specify his or her health level $h_i\in[0, 100]$, the
      coordinate $(x_i, y_j)$, and the task $t_j$ carried by
      the responder. $S_{t_j} = \langle st_j, (x_j, y_j)
      \rangle$ is the state variable for task $t_j$ to specify
      its status $st_j$ (picked up, dropped off, or idle) and
      coordinate $(x_j, y_j)$.
  \item $A_i$ is a set of responder $p_i$'s actions. Each
      responder can {\em stay} in the current location $(x_i,
      y_i)$, {\em move} to the 8 neighbouring locations, or
      {\em complete} a task located in $(x_i, y_i)$. A joint
      action $\vec{a}=\langle a_1, \cdots, a_n \rangle$ is a
      set of actions where $a_i\in A_i$, one for each
      responder.
  \item $P = P_r \times P_{p_1} \times P_{p_n} \times P_{t_1}
      \times P_{t_n}$ is the transition function.
      $P_r(s'_r|s_r)$ is the probability for the radioactive
      cloud to spread from state $s_r$ to $s'_r$. It caputers
      the uncertainty of the next radioactive levels of the
      environment due to the noisy sensor reading and the
      variation in wind speed and direction.
      $P_{p_i}(s'_{p_i}|s, a_i)$ is the probability for
      responder $p_i$ to transit to a new state $s'_{p_i}$ when
      executing action $a_i$. For example, when a responder is
      asked to go to a new location, he or she may not be there
      because he or she becomes tired, gets injured, or
      receives radiation doses that are life threatening.
      $P_{t_j}(s'_{t_j}|s, \vec{a})$ is the probability for
      task $t_j$. A task $t_j$ can only be completed by a team
      of responders with required roles locating in the same
      coordinate as $t_j$.
  \item $R$ is the reward function. If a task is completed, the
      team will be rewarded. There will be a penalty for the
      team if any responder gets injured or receives too many
      radioactive doses. Each action of the responders has a
      cost since it will consume energy of the responders.
\end{itemize}
A policy for the MMDP is a mapping from states to joint actions,
$\pi: S \rightarrow \vec{A}$ so that the responders know which
actions to take given the current state of the problem. The quality
of a policy $\pi$ is usually measured by its expected value
$V^\pi$, which can be computed recursively by the Bellman equation:
\begin{equation}
  V^\pi(s) = R(s, \pi(s)) + \gamma\sum_{s'\in S} P(s'|s, \pi(s)) V^\pi(s')
\end{equation}
where $\pi(s)$ is a joint action given $s$ and $\gamma\in(0, 1]$ is
the discounted factor. The goal of solving the MMDP is to find an
optimal policy $\pi^*$ that maximises the expected value with the
initial state $s^0$, $\pi^* = \arg\max_{\pi} V^\pi(s^0)$.

At each decision step, we assume the planning agent can fully
observe the state of the environment $s$ by collecting sensor
reading of the radioactive cloud, GPS data of the responders, etc.
Given a policy $\pi$ of the MMDP, a joint action $\vec{a}=\pi(s)$
can be selected and broadcasted to the responders (as mentioned
earlier). By so doing, each responder can be instructed by the
agent and know how to act in the field.

