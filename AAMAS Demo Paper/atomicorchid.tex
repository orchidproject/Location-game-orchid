\section{The A\lowercase{tomic}O\lowercase{rchid} Game}\label{sec:atomicorchid}
\noindent We adopt a serious mixed-reality games approach to counteract the limitations of computational simulations \cite{Fischer:etal:2012}. The impact of emotional and physical responses likely in a disaster, such as stress, fear, exertion or panic remains understudied in approaches that rely purely on computational simulation \cite{drury:etal:2009}. In contrast, our approach creates a realistic setting in the sense that participants experience physical exertion and stress through bodily activity and time pressure, mirroring aspects of a real disaster setting \cite{paho:2001}. This, in turn, provides greater confidence in the efficacy of behavioural observations regarding team coordination supported by a planning agent.
\subsection{Interacting with the Planning Agent}
\noindent To permit the interaction between First Responders and the planning agent, the former are equipped with a `mobile responder tool' providing sensing and awareness capabilities in three tabs (geiger cou\-nter, map, messaging and tasks; see Figure \ref{fig:ui}). The first tab shows a reading of radioactivity, player health level (based on exposure), and a GPS-enabled map of the game area to locate fellow responders, the targets to be rescued and the drop off zones for the targets. The second tab provides a broadcast messaging interface to communicate with fellow first responders and the commander $H$. The third tab shows the team and task allocation dynamically provided by the agent $PA$ that can be accepted or rejected. Notifications are used to alert both to new messages and task allocations.

\subsection{Running the Demo}
\noindent For the purposes of a demo at AAMAS, it was envisaged that conference attendees will be recruited as players of the game during the demo session and that the game will be run live. As it stands, the venue for the conference is in a location where there will be poor GPS reception and lack of space for a live game to take place. We will therefore endeavour to run sessions of the game in the Garden of the Observatory (Boulevard de l'Arago)  (7 mins walk from the conference centre) provided approval is granted for this. Otherwise, we will run videos of the demo and allow attendees to play with the mobile app and interact with the planning agent in simulated scenarios.


 %AtomicOrchid is based on the open-sourced geo-fencing game MapAttack\footnote{\url{http://mapattack.org}.} that has been iteratively developed for a responsive, (relatively) scalable experience.  The location-based game is underpinned by client-server architecture, that relies on real-time data streaming between client and server. Client-side requests for less dynamic content use HTTP. Frequent events (e.g., location updates and radiation exposure) are streamed to clients to avoid the overhead of HTTP. In this way, first responders are kept informed in near real-time. Finally,  to build the mobile app, we adapted the existing MapAttack Android app.


\begin{figure}[htbp]
\includegraphics[width=\columnwidth]{UI.png}
\caption{Mobile first responder and HQ interfaces.\vspace{-3mm}
\label{fig:ui}
}\end{figure}

For live runs, the headquarters will be located in the conference centre. $H$ will have at her disposal an `HQ dashboard' that provides an over\-view of the game area, including real-time information of the players' locations (see Figure \ref{fig:ui}). The dashboard provides a broadcast messaging widget, and a player status widget so that the responders' exposure and health levels can be monitored. $H$ can further monitor the  current team and task allocations to individual responders by $PA$ (by clicking on a button). Crucially, only $H$ and $PA$ have a view of the radioactive cloud, graphically depicted as a heatmap (`Hotter'  (red) zones correspond to higher levels of radiation).

First responders will be assigned a specific type: medic, fire-fighter, soldier, or transporter. Their mission is to evacuate all four types of targets: victim (requires medic and fire-fighter), animal (requires medic and transporter), fuel (requires soldier and fire-fighter), or other resource (requires soldier and transporter).  The first responders are supported by (at least) one person $H$ in a centrally located HQ room, and the planning agent $PA$ that sends the next task (as described in the previous section) to the team of first responders. 

The game will last for a maximum 30 mins. Participants will be briefed and asked to consent to participate. They will be presented with a demographic questionnaire to record gender, occupation, experience of using smartphones and level of map navigation skills. At the end of the briefing in which mission objectives and rules will be outlined, responder roles will be randomly assigned to all participants (fire-fighter, medic, transporter, soldier).






%\subsection{Radiation Cloud Modelling}
The radiation cloud diffusion process is modelled by a nonlinear Markov field stochastic differential equation, which assumes the cloud intensity is Gaussian distributed in log-space.  The cloud is driven by wind forces which vary both spatially and temporally.  Wind forces induce an anisotropic diffusion coefficient into the cloud diffusion process.  The wind velocity is modelled by two a priori independent Gaussian processes (GP), one GP for each Cartesian coordinate axis.  The GP captures both the spatial distribution of the wind velocity and also the dynamic process resulting from shifting wind patterns such as short term gusts and longer term variations.  In our simulation, each spatial wind velocity component is modelled by a squared-exponential GP covariance function, $K$, with fixed input and output scales over time (although any covariance function, stationary or not, can be substituted).

Both the radiation cloud and wind model priors are combined into a single joint model called a {\it latent force model} (LFM)~\cite{alvarez09} and predictions of the radiation cloud intensity are inferred using the extended Kalman filter (EKF).  The EKF provides both the mean and variance of the log-radiation cloud intensity and wind conditions.  Uncertainty arises due to unknown initial conditions of the cloud and wind conditions and it is also induced by the stochastic nature of their processes.  The EKF state $S(t)=(\underline{R}(t) \underline{V}_x(t) \underline{V}_y(t))^T$ represents both Cartesian components of the wind velocity, $V_x(t)$ and $V_y(t)$, and the log-radiation cloud density, $R(t)$, on a regular $N\times M$ grid defined across the environment with grid coordinates $G$.  The temporal component of the wind GP model is assumed Markovian and thus, the wind dynamics are incorporated within the EKF as per the KFGP~\cite{reece10}.  For example, the $N\times M$ x-component of the wind velocity at time-step $t+1$ is $V_x(t+1)=F V_x(t)+\nu_t$, where the process model $F=\rho I$ (where $I$ is the identity matrix) and Gaussian process noise $\nu_t\sim \Bbb{N}(0,(1-\rho^2) K(G,G))$ for correlation, $\rho$, of the wind field between time steps.  When $\rho=1$ the wind velocities are time invariant (although spatially variant).  Values of $\rho<1$ model wind conditions that change over time.

The cloud intensity and wind velocity are measured by {\it monitor agents} equipped with geiger-counters and anemometers.  These agents are directed to take measurements with greatest information gain in the radiation cloud intensity.  The measurements are folded into the EKF and this refines estimates of the radiation cloud across the grid.  Figure~\ref{radiation_screen_shots} shows example cloud simulations for invariant (i.e. $\rho=1.0$) and gusty (i.e. $\rho=0.90$) wind conditions.  Figure~\ref{radiation_screen_shots}(a) shows invariant wind conditions in which case the radiation cloud can be interpolated accurately using sparse sensor measurements and the LFM model.  Alternatively, during gusty conditions the radiation cloud model is more uncertain far from the locations where recent measurements have been taken, as shown in Figure~\ref{radiation_screen_shots}(b).

\begin{figure}[ht] \begin{center}
    \includegraphics[width=0.45\textwidth]{figures/radiation_ss_calm.png}\\
    (a) Slow varying wind conditions\\ \ \\
    \includegraphics[width=0.45\textwidth]{figures/radiation_ss_gust.png}\\
    (b) Gusty wind conditions 
\caption{\label{radiation_screen_shots} Radiation and wind simulation ground truth and EKF estimates obtained using measurements from monitor agents (black dots).  Left most panes are ground truth radiation and wind conditions, the middle panes are corresponding estimates and right most panes are state uncertainties:  (a) Invariant and (b) gusty wind conditions.}
\end{center}
\end{figure}

%\section{The Field-Trials}\label{sec:evaluation}
\noindent We ran three sessions of AtomicOrchid with participants recruited from the local university to trial mixed-initiative coordination in a disaster response scenario. The following sections describe the participants, procedure, session configuration and methods used to collect and analyse quantitative and qualitative data.

\subsection{Participants and Procedure}
\noindent  A total of 24 participants (7 of them were female) were recruited through posters and emails, and reimbursed with \pounds 15  for 1.5-2 hours of study. The majority were students of the local university. The procedure consisted of 30 minutes of game play, and about 1 hour in total of pre-game briefing, consent forms,  a short training session, and a post-game group discussion. 

%Upon arrival in the HQ (set up in a meeting room at the local university), participants were briefed and asked to consent to participate. They were presented with a demographic questionnaire to record gender, occupation, experience of using smartphones and level of map navigation skills.

At the end of the briefing, in which mission objectives and rules were outlined, responder roles were randomly assigned to all participants (fire-fighter, medic, transporter, soldier). The HQ was staffed by a different member of the research team in each session in order to mimic an experienced $H$ whilst avoiding the same person running the HQ every time.  Moreover, responders were provided with a smartphone and $H$ with a laptop. The team was given 5 minutes to discuss a common game strategy. 

%(\textbf{Joel: where did the agent run ? --> Gopal, this should be covered in the previous section I think?})

First responders were then accompanied to the starting point within the designated game area, about 1 minute walk from headquarters. Once first responders were ready to start, $H$ sent a `game start' message. After 30 minutes of game play the first responders returned to the HQ where a group interview was conducted, before participants were debriefed and dismissed.

\subsection{Game Sessions}
\noindent We ran one session without $PA$, and two sessions with $PA$ to be able to compare team performance in the two versions. Each session involved \emph{different} sets of players (8 each), that is, players were unique to a session to avoid learning effects between sessions. We also ran a pilot study for each condition to fine tune game configuration. The 8 first responders in each session were randomly allocated a role so that the whole team had two of each of the four kinds of responder roles. The terrain of the 400x400 metre  game area includes grassland, a lake, buildings, roads,  footpaths, and lawns. There were two safe drop-off zones (i.e., radiation free) and 16 targets in each session. There were four targets for each of the four target types. The target locations, pattern of cloud movement and expansion were kept constant for all game sessions. The pilot study showed that this was a challenging, yet not too overwhelming configuration of game area size, and number of targets to collect in a 30 min game session. 

\subsection{Data Collection and Analysis}
\noindent We developed a log file replay tool to triangulate video recordings of game action with the timestamped system logs that contain a complete record of the game play, including responders' GPS location, their health status and radioactive exposure, messages, cloud location, locations of target objects and task status.

%Video recordings of field action were catalogued to identify sequences (episodes) of interest (cf. Heath et al., 2010). Key decision points in teaming and task allocation served to index the episodes. Interesting distinct units of interaction were transcribed and triangulated with log files of relevant game activity for deeper analysis. Due to space constraints we can only  present one fragment in this paper to illustrate how human-agent collaboration typically unfolded (TODO).

In order to assess how humans interact with each other and with $PA$, we focused on collecting data relevant to agent task allocations and remote messages  that are used to support coordination. In particular, we use speech-act theory \cite{searle:1975} to classify messages sent between and among responders and $H$. We focus on the most relevant types of acts in this paper (which are also the most frequently used in AtomicOrchid):

\begin{itemize}
\item Assertives: \textit{speech acts that commit a speaker to the truth of the expressed proposition}; these were a common category as they include messages that contain situational information.
\item Directives: \textit{speech acts that are meant to cause the hearer to take a particular action}, e.g. requests, commands and advice, including task and team allocation messages. 
\end{itemize}

\subsection{Results}
\noindent Overall, 8 targets were rescued in the non-agent condition (Session A), and respectively 12 targets (Session B), and 11 targets (Session C) were rescued in the agent condition. Teams (re-)formed six times in session A, four times in session B and nine times  in session C. Average player health after the game was much higher (more than double) for the agent-assisted sessions (80 for Session B and 82 for Session C) compared to the non-agent assisted session (40/100 in Session A). In fact, one responder `died' in Session A.
$PA$ dynamically re-planned 14 times in session B, and 18 times in session C. In most cases, this was triggered when a target was dropped off in the safe zone (24 times) -- as this would free up resources for the algorithm to recompute an allocation -- in the remaining cases this was triggered by a player rejecting the agent's task allocation (8 times). 

Table \ref{tab:msgs} shows the remote message directives (mainly related to task allocation and execution) and assertives (mainly related to situational awareness) sent in the sessions. The next sections draw on how these messages were handled to give a sense of mixed-initiative coordination in the game sessions.

\begin{table}\footnotesize\small
\begin{tabular}{c | c c | c c c c | c}
 & \multicolumn{2}{c|}{no agent} &  \multicolumn{4}{c|}{agent} & Total \\
 \hline
 Speech acts & \multicolumn{2}{c|}{Session A} & \multicolumn{2}{c}{Session B} & \multicolumn{2}{c|}{Session C} & \\
  & HQ & FR & HQ & FR & HQ & FR & \\
  \hline
  Directives & 89 & 0 & 34 & 2 & 34 & 0 & 159 \\
  Assertives & 33 & 6 & 26 & 16 & 24 & 16 & 121 \\
  \hline
  Total & 122 & 6 & 60 & 18 & 58 & 16 & 280 \\
\end{tabular}

 \caption{Message classification.} \label{tab:msgs}
\end{table}


%\subsubsection{Handling Task Allocations}
\noindent Figure \ref{fig:msgs} shows how first responders handled task allocations in the agent and in the non-agent condition. In the non-agent condition, the HQ commander sent 43 task allocation directives. Out of these, the recipient first responders addressed only 15 messages (bringing them up in conversation). Out of these 15, responders chose to ignore the instructions only once. The responders ignored the instruction because they were engaged in another task and did not want to abandon it. A further 4 $H$ instructions were consistent with a decision to rescue a certain target that had already been agreed locally by the responders. In the remaining 10 cases, first responders chose to follow the instructions. Although players were willing to follow $H$'s instructions, they failed to correctly follow the instructions due to confusion and misunderstanding in the communication. In fact, only 2 instances of directives from $H$ led to task completion. The first responders performed tasks selected for the other 6 saved targets locally without being instructed by $H$.

In contrast, when task allocation was handled by the agent (52 tasks allocated in two trials on average), responders explicitly accepted 24 tasks, out of which they completed 15 tasks successfully. There was no response nor consensus between the responders (in 17 tasks allocated), although 6 out of 17 tasks were completed successfully. In total, 20 task allocations were withdrawn by the agent as a result of re-planning. 

\begin{figure}[htbp]
\includegraphics[width=\columnwidth]{message_handling.png}
\vspace{-3mm}
\caption{How task allocations were handled by first responders in the agent version (left) and in the no-agent version (right).\vspace{-3mm}}\label{fig:msgs}
\end{figure}


%\paragraph{Rejecting task allocations}
In terms of task rejections, first responders rejected $PA$'s task allocation 11 times in the agent version. All of the rejections happened when the task allocation would have \emph{split existing teams}, or instructed responders to team up with \emph{physically more distant responders}. In most cases (9 out of 11), they triggered re-planning by rejection and \emph{adjusted the task allocation} to become consistent with the responder's current team. In the other two cases, the responders rejected  the task allocation one more time before receiving the desired task allocation. For accepted instructions, the average distance between suggested teammates was 12 metres. For rejected instructions, the average distance between suggested teammates was 86 metres.

The results above show that  the simple mechanism to get $PA$ to re-plan (i.e., reject/accept) was far more successful (more tasks completed and less confusion) than the open-ended interactions between $H$ and the responders (that were open to confusion).  Moreover, the fact that many of the rejections were due to the long distance to travel and teammate preference, implies that players chose to do the tasks they \emph{preferred}  rather than those deemed optimal by the agent. This indicates there may be an issue of trust in the agent but also that it may be easier for a responder  to impose (through a reject) such preferences on an agent (and indirectly to other team members) rather than expressing this to $H$ or directly to teammates. 
% Role of HQ commander
It is also important to note that in the agent-assisted setting, $H$ frequently \emph{monitored} the allocation of tasks  returned by the agent (57 clicks on `show task' in UI responder status widget). Whereas 43 directives out of 68 in the non-agent session were task allocations, only 16 out of 68 were directly related to task allocations in the agent version. Out of these, $H$ directly reinforced the agent's instruction 6 times (e.g., ``SS and LT retrieve 09''), and complemented (i.e., added to or elaborated) $PA$'s task allocation 5 times (e.g., ``DP and SS, as soon as you can head to 20 before the radiation cloud gets there first''). $H$  did `override' $PA$'s instruction in 5 cases.  

In the agent version, the majority of $H$'s directives (52 out of 68) and assertives (49 out of 51) focussed on providing situational awareness and safely routing the responders to avoid exposing them to radiation. For example, ``Nk and JL approach drop off 6 by navigating via 10 and 09.'', or ``Radiation cloud is at the east of the National College''. 

%\subsubsection{Summary and Guidelines}\label{sec:summary}
In general, these results suggest three key observations with regard to  human-agent coordination in the trial:\vspace{-2mm}
\begin{itemize}
\item First responders performed better (rescued more targets), and maintained higher health levels when supported by the agent.  These results echo those obtained under simulation (see Section \ref{sec:algo}) and  may reflect the better forward-planning capability of the planning agent compared to human responders. 
 
\item Rejecting tasks was relatively frequently employed to trigger re-planning to obtain new task allocations aligned with responder preferences.  In each case the planning agent was able to adapt to provide an alternative plan that was acceptable to the responders. Without this facility we fear that the responders would have chosen to ignore the plan. Task rejection seemed to be linked to changes to established teams, especially when members were relatively distant. Consequently, these kinds of allocations may need particularly support (e.g., explanation), or might be less preferentially selected by $PA$.

\item When task allocation was handled by $PA$, $H$ could focus on providing vital situational awareness to safely route first responders around danger zones; thus demonstrating division of labour and complementary collaboration between humans and agents.
\end{itemize}

Given the above results we argue that a planning agent for team formation should not only model the uncertainty in player behaviours and the environment, but that interactional challenges also need to be addressed  if such a technology is to be accepted in practice. In particular, we propose the following design guidelines for human-agent collaborations:\\

\noindent \textbf{Adaptivity}: our experience suggest that planning algorithms should be designed to take in human input, and more importantly, be \emph{responsive} to the needs of the users. As we saw in AtomicOrchid, players repeatedly requested new tasks and this would not have been possible unless the algorithm we had designed was computationally efficient but also had the ability to assimilate updates, requests, and constraints dynamically. We believe this makes the algorithm more acceptable.\\

\noindent \textbf{Interaction Simplicity}: our agent was designed to issue simple commands (Do X with Y) and respond to simple requests (OK or Reject Task). Such simple messages were shown to be far more effective at guiding players to do the right task than the unstructured human communication in the non-agent assisted case that was fraught with inconsistencies and inaccuracies. In fact, we would suggest that agents should be designed with minimal options to simplify the reasoning users have to do to interact with the agent, particularly when they are under pressure to act.\\

\noindent \textbf{Flexbile autonomy}: the HQ dashboard proved to be a key tool for the HQ coordinator $H$ to \emph{check} and \emph{correct for} the allocations of $PA$, taking into account the real-world constraints that the players on the ground faced. In particular, letting the human oversee the agent (i.e., ``on-the-loop") at times and actively instructing  the players (and bypassing the agent) at other times (i.e., ``in-the-loop") as and when needed, was seen to be particularly effective. This was achieved by $H$ \emph{without the agent} defining when such transfers of control should happen (as in \cite{scerri:etal:2005}) and, therefore left the coordinator the option of taking control when she judged it was needed. Hence, we suggest that such deployed autonomous systems should be built for flexible autonomy, that is, interfaces should be designed to pass control \emph{seamlessly} between human and agents.





%\section{Real-world trial}
%We ran three sessions of AtomicOrchid with participants recruited from the local university to trial mixed-initiative coordination in a disaster response scenario. 
%
%\subsection{Participants and procedure}
%A total of 24 participants (7 of them were female) were recruited through posters and emails, and reimbursed with 15 pounds for 1.5-2 hours of study. The majority were students of the local university.
%
%The procedure consisted of 30 minutes of game play, and about 1 hour of pre-game briefing, consent forms and a short training session, and a post-game group discussion. 
%
%%Upon arrival in the HQ (set up in a meeting room at the local university), participants were briefed and asked to consent to participate. They were presented with a demographic questionnaire to record gender, occupation, experience of using smartphones and level of map navigation skills.
%
%At the end of the briefing in which mission objectives and rules were outlined, responder roles were randomly assigned to all participants (fire-fighter, medic, transporter, soldier). HQ was staffed by a different member of the research team in each session in order to mimick an experienced HQ whilst avoiding the same person running HQ every time. 
%
%Field responders were provided with a smartphone; HQ coordinators with a laptop. The team was given 5 minutes to discuss a common game strategy. 
%
%%(\textbf{Joel: where did the agent run ? --> Gopal, this should be covered in the previous section I think?})
%
%Field responders were then accompanied to the starting point within the designated game area, about 1 minute walk from headquarters. Once field responders were ready to start, HQ sent a `game start' message. After 30 minutes of game play the field responders returned to the HQ where a group interview was conducted, before participants were debriefed and dismissed.
%
%\subsection{Game sessions}
%We ran one session without the planner agent, and two sessions with the planner agent to be able to compare team performance in the two versions. We also ran a pilot study for each condition to fine tune game configuration. The 8 field responders in each session were randomly allocated a responder so that the whole team had two of each of the four kinds of responder roles. The terrain of the 400x400metres game area includes grassland, a lake, buildings, roads, and footpaths and lawns. There were two drop off zones and 16 targets in each session. There were four targets for each of the four target types. The target locations, pattern of cloud movement and expansion were kept constant for all game sessions. The pilot study showed that this was a challenging, yet not too overwhelming configuration of game area size, and number of targets to collect in a 30 min game session. 
%
%\subsection{Data collection and analysis}
%We developed a log file replay tool to triangulate video recordings of game action with the time stamped system logs that contain a complete record of the game play, including responders' GPS location, their health status and radioactive exposure, messages, cloud location, locations of target objects and task status.
%
%%Video recordings of field action were catalogued to identify sequences (episodes) of interest (cf. Heath et al., 2010). Key decision points in teaming and task allocation served to index the episodes. Interesting distinct units of interaction were transcribed and triangulated with log files of relevant game activity for deeper analysis. Due to space constraints we can only  present one fragment in this paper to illustrate how human-agent collaboration typically unfolded (TODO).
%
%In this paper we focus on how both agent task allocations and remote messages are used as a coordination resource. We use speech-act theory (Searle, 1975) to classify messages sent between and among responders and HQ. We focus on the most relevant types of acts in this paper (which are also the most frequently used in AtomicOrchid):
%
%\begin{itemize}
%\item Assertives: \textit{speech acts that commit a speaker to the truth of the expressed proposition}; these were a common category as they include messages that contain situational information.
%\item Directives: \textit{speech acts that are meant to cause the hearer to take a particular action}, e.g. requests, commands and advice, including task and team allocation messages. 
%\end{itemize}
%
%\subsection{Results}
%Overall, 8 targets were rescued in the non-agent condition (Session A), and respectively 12 targets (Session B), and 11 targets (Session C) were rescued in the agent condition. Teams (re-)formed six times in session A, four times in session B and nine times  in session C. Average player health after the game was 40/100 in Session B, and 81 for the agent-assisted sessions (B:80, C: 82). One responder `died' in Session A.  
%
%The agent dynamically re-planned 14 times in session B, and 18 times in session C. Most of the times, this was triggered when a target was dropped off in the safe zone (24 times), some times this was triggered by a player rejecting the agent's task allocation (8 times). 
%
%Table \ref{tab:msgs} shows the directives (mainly related to task allocation and execution) and assertives (mainly related to situational awareness) sent in the sessions. The next sections draw on how these messages were handled to give a sense of mixed-initiative coordination in the game sessions.
%
%\begin{table}\footnotesize
%\begin{tabular}{c | c c | c c c c | c}
% & \multicolumn{2}{c|}{no agent} &  \multicolumn{4}{c|}{agent} & Total \\
% \hline
% Speech acts & \multicolumn{2}{c|}{Session A} & \multicolumn{2}{c}{Session B} & \multicolumn{2}{c|}{Session C} & \\
%  & HQ & FR & HQ & FR & HQ & FR & \\
%  \hline
%  Directives & 89 & 0 & 34 & 2 & 34 & 0 & 159 \\
%  Assertives & 33 & 6 & 26 & 16 & 24 & 16 & 121 \\
%  \hline
%  Total & 122 & 6 & 60 & 18 & 58 & 16 & 280 \\
%\end{tabular}
% \label{tab:msgs}
% \caption{Message classification.}
%\end{table}
%
%
%\subsubsection{Handling task allocations}
%Fig. \ref{fig:msgs} shows how field responders handled task allocations in the agent and in the non-agent condition. In the non-agent condition, HQ sent 43 task allocation directives. Out of these, the recipient field responders addressed only 15 messages (bringing them up in conversation). Out of these 15, responders chose to ignore the instructions only once. The responder ignored the instruction because they were engaged in another task and did not want to abandon it. A further 4 HQ instructions were consistent with a decision to rescue a certain target that has already been made locally by the responders. In 10 cases field responders chose to follow the instructions. Although players were willing to follow HQ's instructions, they failed to correctly follow the instructions due to confusion and misunderstanding in the communication. In fact, only 2 instances of directives from the HQ led to task completion.The field responders accomplished task allocation of the other 6 saved targets locally without being instructed by HQ.
%
%On the other hand, when task allocation was handled by the agent, responders accepted 24 tasks, out of which they completed 15 tasks successfully. Even if there was no response or consensus between the responders (in 17 cases), still six out of 17 tasks were completed successfully. In total, 20 task allocations were overridden by a the agent with a new task allocation. 
%
%\begin{figure}[htbp]
%\includegraphics[width=\columnwidth]{message_handling.png}
%\label{fig:msgs}
%\caption{How task allocations were handled by field responders in the agent version (left) and in the non-agent version (right).}
%\end{figure}
%
%\paragraph{Rejecting task allocations}
%
%Field responders rejected the agent's task allocation 11 times in the agent version. All of the rejections happened when the task allocation would have split existing teams, or instructed responders to team up with physically more distant responders. In most cases (9 out of 11), the rejection triggered re-planning and adjusted the task allocation to become consistent with the responder's current team. In the other 2 cases, rejected the task allocation one more time to receive the desired task allocation. 
%
%Overall, players were more likely to reject plan if their proposed teammates were far away from them. For accepted instructions, the average distance between suggested teammates was 12 metres. For rejected instructions, the average distance between suggested teammates was 86 metres.
%
%\subsubsection{The role of HQ}
%The role of HQ changed in that in the non-agent version HQ was responsible for handling task allocations, whereas task allocation was handled by the agent in the other version. HQ frequently monitored the agent's task allocation in the two agent-supported sessions (57 clicks on `show task' in UI responder status widget). Whereas 43 directives in the non-agent session were task allocations, only 16 out of 68 directives were directly related to task allocations in the agent version. Out of these, HQ reinforced the agent instruction 6 times (e.g., ``SS and LT retrieve 09''), and complemented the agent's task allocation 5 times (``DP and SS, as soon as you can head to 20 before the radiation cloud gets there first''). HQ did `override' the agent instruction in 5 cases.  
%
%In the agent version, the majority of HQ's directives (52 out of 68) and assertives (49 out of 51) focussed on providing situational awareness and safely routing the responders to avoid exposing them to radiation. For example, ``Nk and JL approach drop off 6 by navigating via 10 and 09.'', or ``Radiation cloud is at the east of the National College''. 
%
%\subsubsection{Summary}
%The results suggest three key observations with regard to mixed-initiative coordination in the trial:
%
%\begin{itemize}
%\item Field responders performed better (rescued more targets), and maintained higher health levels when supported by the agent.
%\item Rejecting tasks was a frequently employed method to to trigger re-planning to obtain new task allocations aligned with responder preferences.
%\item When task allocation was handled by the agent, the human HQ could focus on providing vital situational awareness to safely route field responders around danger zones; thus demonstrating division of labour and complementary collaboration between human and agent.
%\end{itemize}
%
% 
%%Joel and Wenchao
%%\begin{enumerate}
%%\item Explain setup of experiment - area of interest + setup of tasks
%%\item Explain evaluation = quantitative and qualitative.
%%\end{enumerate}
%%\paragraph{Metrics}
%%\begin{itemize}
%%\item{Comparisons between with/without agent versions for the below:}
%%\item{Performance of FR: number of tasks completed, time on task?, number of messages sent, number of teams formed and disbanded, time on team, acknowledgements of tasks}
%%\item{Messages: classification}
%%\item{Health}
%%\item{Distance travelled}
%%\item{HQ: number of agent monitoring actions (clicks), number of 'supporting'/related messages (e.g., enforcement, contradictions/overriding)}
%%\item{Agent performance: number of instructions, number of replanning steps, replanning robustness (diversion of task allocation compared to previous step)}
%%\item{Following instructions ('obedience'): number of instructions followed vs. not followed (incl. number of HQ interventions/overriding agent allocation), instruction handling diagram}
%%\item
%%\end{itemize}
