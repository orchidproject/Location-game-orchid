Given a reasonable size of the game, the MMDP model can be huge.
For example, with 8 players and 17 tasks in a 50$\times$55 grid,
the number of possible states is more than $2\times 10^{400}$.
Thus, it is computational intractable to find the optimal plan of
the game. One useful observation is that, when making a decision,
each player first selects a task to form a team and then move to
the task location independently. In our planning algorithm, we use
this hierarchical structure to decompose the decision making
process into two level:
\begin{itemize}
  \item In the higher level, task planning algorithm is run for
      the whole team to assign the best task to each players
      given the current state.
  \item In the lower level, by given a task, path planning
      algorithm is run for each player to find the best path to
      the task from his current location.
\end{itemize}
Given the huge state space, it is intractable to find the plan for
all states offline. Therefore, we only compute plans for the
queried states online. The main process of the online hierarchical
planning algorithm is outlined in Algorithm~\ref{alg:coordination}.
The following subsections will describe the algorithms in each
level in more detail.

\begin{algorithm}[t]
  \caption{Team Coordination}
  \KwIn{the MMDP model and the current state $s$}
  \KwOut{the best joint action $\vec{a}$}
  \tcp{The task planning}
  $\{ t_i \} \gets$ compute the best task for each player $i\in I$ \;
  \ForEach{$i\in I$} {
    \tcp{The path planning}
    $a_i \gets$ compute the best path to task $t_i$ \;
  }
  \Return{$\vec{a}$}
  \label{alg:coordination}
\end{algorithm}

\subsection{Task planning}
As mentioned above, each player in the game owns a type of skill
and each task requires players with a certain combination of the
skills. The goal is to assign a task to each player that maximize
the overall performance given the current state. To do that, we
first compute all possible coalitions $\{ C_{jk} \}$ for each task
$j$ where a coalition $C_{jk} \subseteq I$ is a subset of the
players with the required skills. Then, we solve the following
optimization problem to find the best coalitions:
\begin{equation}
  \begin{array}{lll}
    \max_{x_{jk}} & \sum_{j, k} x_{jk} \cdot v(C_{jk}) & \\
    \mbox{s.t.} & x_{jk} \in \{0, 1\} & \\
    & \forall j, \sum_{k} x_{jk} \leq 1 & \mbox{(i)} \\
    & \forall i, \sum_{j, k} \delta_i(C_{jk}) \leq 1 & \mbox{(ii)}
  \end{array}
  \label{eq:cf}
\end{equation}
where $x_{jk}$ is the boolean variable to decide whether to select
coalition $C_{jk}$ for task $j$ or not, $v(C_{jk})$ is the
characteristic function for coalition $C_{jk}$, and
$\delta_i(C_{jk}) = 1$ if $i\in C_{jk}$ and 0 otherwise. Constraint
(i) ensures that a task $j$ is allocated at most to only one
coalition (a task does not need more than one group of players).
Constraint (ii) ensures that a player $i$ is assign to only one
task (a player cannot do more than one task at the same time). In
the optimization, we only consider the tasks that have not been
done and the players that are still alive. Because the players do
not change their skills and the requirements of the tasks are
static during the game, the set of all possible coalitions for each
task can be computed offline before the game.

The main challenge of the optimisation problem in
Equation~\ref{eq:cf} is to compute the value of the characteristic
function $v(C_{jk})$ for each coalition $C_{jk}$. This is the
expected value when the players in $C_{jk}$ are assigned to task
$j$. In order to compute this value, we need to know the plan after
the completion of task $j$ because not all tasks can be completed
in one shot. As aforementioned, computing the optimal plan is
intractable. Thus, we estimate the value by simulations.

[[More detail about sampling will be added.]]

\subsection{Path planning}

In the path planning, we compute the best path for a player given
the location of his assigned task. This is a single-agent problem
that can be modeled as a MDP, $\langle S_i, A_i, P_i, R_i \rangle$,
where:
\begin{itemize}
  \item $S_i = S_r \times S_{p_i}$ is the state space. Player
      $i$ only need to consider his own state variable
      (location and health level) and the state variable of the
      radiation cloud.
  \item $A_i$ is the set of actions $i$. Player $i$ only need
      to consider the actions of staying in the same grid or
      moving to the 8 neighboring grids.
  \item $P_i = P_r \times P_{p_i}$ is the transition function.
      Player $i$ only need to consider the expanding of the
      radiation cloud and the change of his location and health
      level when moving in the grid map.
  \item $R_i$ is the reward function. Player $i$ has a small
      cost for moving around and a large penalty for being
      killed by entering the radiation cloud.
\end{itemize}
This process terminates when the location of the assigned task is
reached or the player is killed (the health level is 0) by the
radiation cloud. This is a typical MDP and can be solved by
state-of-the-art MDP solvers.

[[More detail about solving MDP will be added.]]
