% This is "aamas2014.tex", a revised version of aamas2013.tex
% This file should be compiled with "aamas2014.cls"
% This example file demonstrates the use of the 'aamas2014.cls'
% LaTeX2e document class file. It is for those submitting
% articles to AAMAS 2014  conference. This file is based on
% the sig-alternate.tex example file.
% The 'sig-alternate.cls' file of ACM will produce a similar-looking,
% albeit, 'tighter' paper resulting in, invariably, fewer pages.
% than the original style ACM style.
%
% ----------------------------------------------------------------------------------------------------------------
% This .tex file (and associated .cls ) produces:
%       1) The Permission Statement
%       2) The Conference (location) Info information
%       3) The Copyright Line with AAMAS data
%       4) NO page numbers
%
% as against the acm_proc_article-sp.cls file which
% DOES NOT produce 1) through 3) above.
%
% Using 'aamas2014.cls' you don't have control
% from within the source .tex file, over both the CopyrightYear
% (defaulted to 200X) and the IFAAMAS Copyright Data
% (defaulted to X-XXXXX-XX-X/XX/XX).
% These information will be overwritten by fixed AAMAS 2014  information
% in the style files - it is NOT as you are used with ACM style files.
%
% ---------------------------------------------------------------------------------------------------------------
% This .tex source is an example which *does* use
% the .bib file (from which the .bbl file % is produced).
% REMEMBER HOWEVER: After having produced the .bbl file,
% and prior to final submission, you *NEED* to 'insert'
% your .bbl file into your source .tex file so as to provide
% ONE 'self-contained' source file.
%

% This is the document class for full camera ready papers and extended abstracts repsectively

\documentclass{aamas2014}

% if you are using PDF LaTex and you cannot find a way for producing
% letter, the following explicit settings may help

\pdfpagewidth=8.5truein
\pdfpageheight=11truein

\usepackage[ruled, vlined]{algorithm2e}
\DontPrintSemicolon

\usepackage{graphicx}

\begin{document}

% In the original styles from ACM, you would have needed to
% add meta-info here. This is not necessary for AAMAS 2014  as
% the complete copyright information is generated by the cls-files.


\title{Mixed-Initiative Coordination for Disaster Response in the Real-World}

% AUTHORS


% For initial submission, do not give author names, but the
% tracking number, instead, as the review process is blind.

% You need the command \numberofauthors to handle the 'placement
% and alignment' of the authors beneath the title.
%
% For aesthetic reasons, we recommend 'three authors at a time'
% i.e. three 'name/affiliation blocks' be placed beneath the title.
%
% NOTE: You are NOT restricted in how many 'rows' of
% "name/affiliations" may appear. We just ask that you restrict
% the number of 'columns' to three.
%
% Because of the available 'opening page real-estate'
% we ask you to refrain from putting more than six authors
% (two rows with three columns) beneath the article title.
% More than six makes the first-page appear very cluttered indeed.
%
% Use the \alignauthor commands to handle the names
% and affiliations for an 'aesthetic maximum' of six authors.
% Add names, affiliations, addresses for
% the seventh etc. author(s) as the argument for the
% \additionalauthors command.
% These 'additional authors' will be output/set for you
% without further effort on your part as the last section in
% the body of your article BEFORE References or any Appendices.

%\numberofauthors{8} %  in this sample file, there are a *total*
% of EIGHT authors. SIX appear on the 'first-page' (for formatting
% reasons) and the remaining two appear in the \additionalauthors section.
%

\numberofauthors{1}

\author{
% You can go ahead and credit any number of authors here,
% e.g. one 'row of three' or two rows (consisting of one row of three
% and a second row of one, two or three).
%
% The command \alignauthor (no curly braces needed) should
% precede each author name, affiliation/snail-mail address and
% e-mail address. Additionally, tag each line of
% affiliation/address with \affaddr, and tag the
% e-mail address with \email.
% 1st. author
\alignauthor
Paper  XXX
%Ben Trovato\titlenote{Dr.~Trovato insisted his name be first.}\\
%       \affaddr{Institute for Clarity in Documentation}\\
%       \affaddr{1932 Wallamaloo Lane}\\
%       \affaddr{Wallamaloo, New Zealand}\\
%       \email{trovato@corporation.com}
% 2nd. author
%\alignauthor
%G.K.M. Tobin\titlenote{The secretary disavows any knowledge of this author's actions.}\\
%       \affaddr{Institute for Clarity in Documentation}\\
%       \affaddr{P.O. Box 1212}\\
%       \affaddr{Dublin, Ohio 43017-6221}\\
%       \email{webmaster@marysville-ohio.com}
% 3rd. author
%\alignauthor Lars Th{\o}rv{\"a}ld\titlenote{This author is the one who did all the really hard work.}\\
%       \affaddr{The Th{\o}rv{\"a}ld Group}\\
%       \affaddr{1 Th{\o}rv{\"a}ld Circle}\\
%       \affaddr{Hekla, Iceland}\\
%       \email{larst@affiliation.org}
}

%\and  % use '\and' if you need 'another row' of author names

% 4th. author
%\alignauthor Lawrence P. Leipuner\\
%       \affaddr{Brookhaven Laboratories}\\
%       \affaddr{Brookhaven National Lab}\\
%       \affaddr{P.O. Box 5000}\\
%       \email{lleipuner@researchlabs.org}

% 5th. author
%\alignauthor Sean Fogarty\\
%       \affaddr{NASA Ames Research Center}\\
%       \affaddr{Moffett Field}\\
%       \affaddr{California 94035}\\
%       \email{fogartys@amesres.org}

% 6th. author
%\alignauthor Charles Palmer\\
%       \affaddr{Palmer Research Laboratories}\\
%      \affaddr{8600 Datapoint Drive}\\
%       \affaddr{San Antonio, Texas 78229}\\
%       \email{cpalmer@prl.com}

%\and

%% 7th. author
%\alignauthor Lawrence P. Leipuner\\
%       \affaddr{Brookhaven Laboratories}\\
%       \affaddr{Brookhaven National Lab}\\
%       \affaddr{P.O. Box 5000}\\
%       \email{lleipuner@researchlabs.org}

%% 8th. author
%\alignauthor Sean Fogarty\\
%       \affaddr{NASA Ames Research Center}\\
%       \affaddr{Moffett Field}\\
%       \affaddr{California 94035}\\
%       \email{fogartys@amesres.org}

%% 9th. author
%\alignauthor Charles Palmer\\
%       \affaddr{Palmer Research Laboratories}\\
%       \affaddr{8600 Datapoint Drive}\\
%       \affaddr{San Antonio, Texas 78229}\\
%       \email{cpalmer@prl.com}

%}

%% There's nothing stopping you putting the seventh, eighth, etc.
%% author on the opening page (as the 'third row') but we ask,
%% for aesthetic reasons that you place these 'additional authors'
%% in the \additional authors block, viz.
%\additionalauthors{Additional authors: John Smith (The Th{\o}rv{\"a}ld Group,
%email: {\texttt{jsmith@affiliation.org}}) and Julius P.~Kumquat
%(The Kumquat Consortium, email: {\texttt{jpkumquat@consortium.net}}).}
%\date{30 July 1999}
%% Just remember to make sure that the TOTAL number of authors
%% is the number that will appear on the first page PLUS the
%% number that will appear in the \additionalauthors section.

\maketitle

\begin{abstract}
The problem of allocating emergency responders to rescue tasks is a key application area for agent-based coordination algorithms. However, to date, none of the proposed approaches take into account the uncertainty predominant in disaster scenarios and, crucially, none have been deployed in the real-world in order to understand how humans perform when instructed by an agent. Hence, in this paper, we propose a novel algorithm, using Multi-agent Markov Decision Processes to coordinate emergency responders. More importantly, we deploy this algorithm in a mixed-reality game to help an agent guide human players to complete rescue tasks. In our field trials, our algorithm is shown to improve human performance and our results allow us to elucidate some of the key challenges faced when  deploying of mixed-initiative team formation algorithms. \end{abstract}

% Note that the category section should be completed after reference to the ACM Computing Classification Scheme available at
% http://www.acm.org/about/class/1998/.

\category{H.4}{Information Systems Applications}{Multi-Agent Systems}

%A category including the fourth, optional field follows...
%\category{D.2.8}{Software Engineering}{Metrics}[complexity measures, performance measures]

%General terms should be selected from the following 16 terms: Algorithms, Management, Measurement, Documentation, Performance, Design, Economics, Reliability, Experimentation, Security, Human Factors, Standardization, Languages, Theory, Legal Aspects, Verification.

\terms{Design, Human Factors, Algorithms}

%Keywords are your own choice of terms you would like the paper to be indexed by.

\keywords{Human-Agent Interaction, Coordination, Decision under Uncertainty, Adjustable Autonomy}

\section{Introduction}
\noindent The coordination of first responders in search and rescue missions is a grand challenge for multi-agent systems research \cite{kitano:2001}. In such settings, responders with different capabilities (e.g., fire-fighting or life support) have to form teams in order to perform rescue tasks  (e.g., extinguishing a fire or providing first aid) to minimise  loss of life and costs (e.g., time or money). These tasks may be geographically distributed  and require specific teams  to be completed. Furthermore, uncertainty in the environment (e.g., wind direction or spread of fire) or in the responders' abilities to complete tasks (e.g., some may be tired or get hurt) means that plans are likely to change continually to reflect the prevailing assessment of the situation. 

To address these challenges, a number of algorithms  have been developed to form teams and allocate tasks. For example, \cite{ramchurn:etal:2010,Scerri2005} and \cite{Chapman2009}, devised centralised and decentralised algorithms respectively to allocate rescue tasks to first responders with different capabilities. However, none of these approaches considered the inherent uncertainty in the environment or in the first responders' abilities. Crucially, to date, while all of these algorithms have been shown to perform well in simulations (representing responders as computational entities), none of them have been \emph{trialled} to guide \emph{real} human responders in real-time rescue missions. Thus, it is still unclear whether these algorithms will cope with real-world uncertainties (e.g., social preference), be acceptable to humans (i.e., be intelligible and effective), and actually augment, rather than hinder,  human performance.

Against this background, we develop a novel algorithm for team coordination under uncertainty and evaluate it within a real-world \emph{mixed-reality game} \cite{Fischer:etal:2012} that embodies the simulation of team coordination in disaster response settings. Specifically, our algorithm is used by a software agent to guide human responders in the game to stay clear of a virtual radioactive cloud and complete a number of geo-located rescue tasks in the real world. By so doing, we  study, both quantitatively and qualitatively, the performance of a human-agent collective (i.e., a mixed-initiative team where control can shift between humans and agents)  and the interactions between the different actors in the system. In particular, we  advance the state of the art in the following ways. First, we develop a novel representation for team coordination (i.e., path planning and task allocation) under uncertainty using Multi-agent Markov Decision Processes (MMDP)  \cite{boutilier1996planning}. Moreover, we provide an approximate algorithm to solve the MMDP and show how it  is adaptive to human requests to re-plan task allocations. Second, we present AtomicOrchid, a novel game to evaluate team coordination under uncertainty using the concept of mixed-reality games. AtomicOrchid allows a planning agent, using our task planning algorithm, to coordinate, in real-time, human players using mobile phone-based messaging, to complete rescue tasks efficiently. Third, we provide a real-world evaluation of our planning agent in a disaster response scenario in multiple field trials and present both quantitative and qualitative results. 
Our results show, for the first time, how agent-based coordination algorithms for disaster response can be integrated and validated with human teams. Moreover, these results allow us to derive  guidelines for systems involving  human-agent collaboration.  In what follows, we first formalise the disaster response problem as a MMDP and then describe the algorithms to solve the the MMDP. Given this we describe the AtomicOrchid platform and present results of our field trials and discuss our design guidelines for human-agent collaboration.\vspace{-2mm}


%%%%%%%%%%%%%%%%%%%%%%%%%%% OLD TEXT %%%%%%%%%%%%%%%%%%%%%%%%%%%%

%In more detail, we consider a scenario involving rescue tasks distributed in a physical space over which a (virtual) radioactive cloud is spreading. Tasks need to be completed by the responders before the area is completely covered by the cloud (as responders will die from radiation exposure) which is spreading according to varying wind speed and direction. Our algorithm captures the uncertainty in the scenario (i.e., in terms of environment and player states) and  is able to compute a policy to allocate responders to tasks that minimises task completion time and ensures responders are not exposed to significant radiation. The algorithm is then used by an agent to guide human responders. Specifically


%The rest of this paper is structured as follows. First we \ref{sec:scenario} formalises the disaster response problem as a MMDP. Section \ref{sec:algo}  describes the algorithms to solve the the MMDP while Section \ref{sec:atomicorchid}  details the AtomicOrchid platform. Section \ref{sec:evaluation} presents our field trials and Section \ref{sec:conclusions} concludes.
\section{The Disaster Scenario}

\noindent We consider a disaster scenario involving a satellite, powered radioactive fuel, that has crashed in a sub-urban area (see Section \ref{atomic} to see how this helps implement a  mixed-reality game). While debris is strewn around a large area, damaging buildings and causing accidents and injuring civilians, radioactive particles discharged in the air, from the debris, are gradually spreading over the area, threatening to contaminate food reserves and people. Hence, emergency services, voluntary organisations, and the military are deployed to help evacuate the casualties and resources before these are engulfed by  radioactive cloud.  In what follows, we model this scenario formally and then describe the optimisation problem faced by the actors  (i.e., including emergency services, volunteers, medics, and soldiers) in trying to save as many lives and resources as possible.  We then propose an algorithm to solve this optimisation problem (in Section \ref{sec:algo}. In Section \ref{sec:atomic}, we show how this algorithm can be used by a software agent (in our mixed-reality game) in a mixed-initiative process to coordinate field responders. 

\subsection{Formal Model}
\noindent Let $G$ denote a grid overlaid on top of the disaster space, and the satellite and actors are located at various coordinates $(x,y) \in G$ in this grid. The radioactive cloud induces a radioactivity level  $l \in [0,100]$ at every point it covers in the grid (100 corresponds to maximum radiation). While the exact radiation levels can be measured by responders on the ground (at every grid coordinate) using their geiger counter, it is assumed that some information is available  from existing sensors  in the area. However, this information is uncertain due to the poor positioning of the sensors and the variations in wind speed and direction (and we show how this uncertainty is captured in the next section). Let the set of field responders be denoted as $i_1, \cdots, i_n \in I$ and the set of rescue tasks as  $t_1,\cdots, t_m\in T$.  As responders enact tasks, they may become tired, get injured, or receive radiation doses that may, at worst, be life threatening. Hence, we assign each responder  a health level $h_i\in [0,100]$ that decreases based on their radiation dose and assume that their decision to perform the task allocated to them is liable to some uncertainty (e.g., they may not want to do a task because they are tired or don't believe it is the right one to do). Moreover, each responder will have  a specific role  $r \in Roles$ (e.g., fire brigade, soldier, or medic) and this will determine the capabilities he or she has and therefore the tasks he or she can perform. We denote as $Roles(i)$ the role of responder $i$. In turn, to complete a given task $t$,  a set of responders $I' \subseteq I$ with specific roles $R_t \subseteq R$ is required. Thus, a task can only be completed by a team of responders $I'$ if $\{Roles(i) | i \in I'\} = R_t$. 

Given this model, we next formulate the optimisation problem faced by the responders (and later solved by the planning agent in Section \ref{sec:algo}). To this end, we propose a Multi-Agent Markov Decision Process (MMDP)~\cite{?} that captures the uncertainties of the radioactive cloud and the responders' behaviours. Specifically, we model the spreading of the radiative cloud as a random process over the disaster space and allow the actions requested from the responders to  fail (because they refuse to go to a  task) or incur delays (because they are too slow) during the rescue process. This stands in contrast to previous work \cite{csftp,HTSSC} that require the process of task executions to be deterministic and explicitly model the task deadlines as deterministic constraints (which are stochastic in our domain). Thus in the MMDP model, we represent  task executions as stochastic processes of state transitions. Thus, the uncertainties of the radioactive cloud and the responders' behaviours can be easily captured with transition probabilities. Additionally, modelling the problem as a MMDP enables us to use many sophisticated algorithms that have already been developed in the literature \cite{XX,YY,ZZ}.\textbf{Feng: which one do we use?}


\subsection{The Optimisation Problem}
\noindent A Multi-agent Markov Decision Process (MMDP) is formally
defined as a tuple $\mathcal{M} = \langle I, S, \{A_i\}, P, R
\rangle$, where:
\begin{itemize}
  \item $I$ is a set of $n$ field responders and each responder
      is associated with a unique identifier number $p_i\in I$.
  \item $S = S_r \times S_{p_1} \times \cdots \times S_{p_n}
      \times S_{t_1} \times \cdots \times S_{t_m}$ is the state
      space. $S_r = \{l_{(x,y)}| (x, y) \in G\}$ is the state
      variable of the radioactive cloud to specify the
      radioactive level $l_{(x,y)}\in[0, 100]$ at every point
      $(x, y)\in G$. $S_{p_i} = \langle h_i, (x_i, y_i), t_j
      \rangle$ is the state variable for each responder $p_i$
      to specify his or her health level $h_i\in[0, 100]$, the
      coordinate $(x_i, y_j)$, and the task $t_j$ carried by
      the responder. $S_{t_j} = \langle st_j, (x_j, y_j)
      \rangle$ is the state variable for task $t_j$ to specify
      its status $st_j$ (picked up, dropped off, or idle) and
      coordinate $(x_j, y_j)$.
  \item $A_i$ is a set of responder $p_i$'s actions. Each
      responder can {\em stay} in the current location $(x_i,
      y_i)$, {\em move} to the 8 neighbouring locations, or
      {\em complete} a task located in $(x_i, y_i)$. A joint
      action $\vec{a}=\langle a_1, \cdots, a_n \rangle$ is a
      set of actions where $a_i\in A_i$, one for each
      responder.
  \item $P = P_r \times P_{p_1} \times P_{p_n} \times P_{t_1}
      \times P_{t_n}$ is the transition function.
      $P_r(s'_r|s_r)$ is the probability for the radioactive
      cloud to spread from state $s_r$ to $s'_r$. It caputers
      the uncertainty of the next radioactive levels of the
      environment due to the noisy sensor reading and the
      variation in wind speed and direction.
      $P_{p_i}(s'_{p_i}|s, a_i)$ is the probability for
      responder $p_i$ to transit to a new state $s'_{p_i}$ when
      executing action $a_i$. For example, when a responder is
      asked to go to a new location, he or she may not be there
      because he or she becomes tired, gets injured, or
      receives radiation doses that are life threatening.
      $P_{t_j}(s'_{t_j}|s, \vec{a})$ is the probability for
      task $t_j$. A task $t_j$ can only be completed by a team
      of responders with required roles locating in the same
      coordinate as $t_j$.
  \item $R$ is the reward function. If a task is completed, the
      team will be rewarded. There will be a penalty for the
      team if any responder gets injured or receives too many
      radioactive doses. Each action of the responders has a
      cost since it will consume energy of the responders.
\end{itemize}
A policy for the MMDP is a mapping from states to joint actions,
$\pi: S \rightarrow \vec{A}$ so that the responders know which
actions to take given the current state of the problem. The quality
of a policy $\pi$ is usually measured by its expected value
$V^\pi$, which can be computed recursively by the Bellman equation:
\begin{equation}
  V^\pi(s) = R(s, \pi(s)) + \gamma\sum_{s'\in S} P(s'|s, \pi(s)) V^\pi(s')
\end{equation}
where $\pi(s)$ is a joint action given $s$ and $\gamma\in(0, 1]$ is
the discounted factor. The goal of solving the MMDP is to find an
optimal policy $\pi^*$ that maximises the expected value with the
initial state $s^0$, $\pi^* = \arg\max_{\pi} V^\pi(s^0)$.

At each decision step, we assume the planning agent can fully
observe the state of the environment $s$ by collecting sensor
reading of the radioactive cloud, GPS data of the responders, etc.
Given a policy $\pi$ of the MMDP, a joint action $\vec{a}=\pi(s)$
can be selected and broadcasted to the responders (as mentioned
earlier). By so doing, each responder can be instructed by the
agent and know how to act in the field.



\section{Team Coordination Algorithm}
Feng and Gopal
\begin{enumerate}
\item Feng's algorithm
\item experimental results in simulation - computational performance + no. of tasks completed in simulated settings. If possible, compare against something else.
\end{enumerate}

\noindent Given a reasonable size of our problem, the corresponding
MMDP model can be very large. For example, with 8 players and 17
tasks in a 50$\times$55 grid, the number of possible states is more
than $2\times 10^{400}$. Therefore, it is computationally
intractable to compute the optimal solution. One useful observation
of our problem is: when making a decision, the responders first
need to {\em cooperatively} select a task to form a team with
others. Then they can {\em independently} compute the best path to
the task. In our planning algorithm, we use this observation to
decompose the decision-making process into a hierarchical structure
with two levels:
\begin{itemize}
  \item In the higher level, task planning algorithm is run for
      the whole team to assign the best task to each responders
      given the current state.
  \item In the lower level, by given a task, path planning
      algorithm is run for each responder to find the best path
      to the task from his or her current location.
\end{itemize}

Furthermore, not all states are relevant to the problem (e.g., if a
responder gets injured, he or she is incapable to do any task in
the future and therefore his or her states are irrelevant to other
responders) and we only need to consider the reachable states given
the current state of the problem. Hence, given the current state,
we compute the policy online only for reachable states. This saves
a lot of computation because the size of the reachable states is
usually much smaller than the overall state space. Another
advantage of online planning is that it allows us to tweak the
model as more information is obtained or unexpected events happen.
For example, if the wind becomes stronger, the uncertainty about
the radioactive cloud may increase. If a responder becomes tired,
his or her actions can be less reliable.

The main process of our online hierarchical planning algorithm is
outlined in Algorithm~\ref{alg:coordination}. The following
sections will describe the procedures of each level in more detail.

\begin{algorithm}[t]
  \caption{Team Coordination}
  \KwIn{the MMDP model and the current state $s$.}
  \KwOut{the best joint action $\vec{a}$.}
  \tcp{The task planning}
  $\{ t_i \} \gets$ compute the best task for each responder $i\in I$ \;
  \ForEach{$i\in I$} {
    \tcp{The path planning}
    $a_i \gets$ compute the best path to task $t_i$ \;
  }
  \Return{$\vec{a}$}
  \label{alg:coordination}
\end{algorithm}

\subsection{Task planning}
\label{sec:taskplanning}

\noindent As aforementioned, each responder in our problem has a
specific role to determine which task he or she can perform. A task
can only be completed by a team of responders with the required
roles. Thus, the goal of task planning is to assign a task to each
responder that maximises the team performance given the current
state $s$. To this end, we first compute all possible coalitions
$\{ C_{jk} \}$ for each task $t_j$ where a coalition $C_{jk}
\subseteq I$ is a group of the responders with the required roles.
Apparently, if a task has been completed, we do not need to
consider it any more. If a responder is incapable of performing the
task, he or she will be removed from the coalitions. This
information can be obtained from the state $s$. Because the role of
each responder and the requirement of each task is static, we can
compute all possible coalitions offline. During the online phase,
we only need to filter out the coalitions for completed tasks or
with incapable responders to compute the coalition set $\{ C_{jk}
\}$.

Given the coalition set computed above, we then solve the following
optimisation problem to find the best solution:
\begin{equation}
  \begin{array}{lll}
    \max\limits_{x_{jk}} & \sum_{j, k} x_{jk} \cdot v(C_{jk}) & \\[2pt]
    \mbox{s.t.} & x_{jk} \in \{0, 1\} & \\[2pt]
    & \forall j, \sum_{k} x_{jk} \leq 1 & \mbox{(i)} \\[2pt]
    & \forall i, \sum_{j, k} \delta_i(C_{jk}) \leq 1 & \mbox{(ii)}
  \end{array}
  \label{eq:cf}
\end{equation}
where $x_{jk}$ is the boolean variable to indicate whether
coalition $C_{jk}$ is selected for task $t_j$ or not, $v(C_{jk})$
is the characteristic function for coalition $C_{jk}$, and
$\delta_i(C_{jk}) = 1$ if responder $p_i\in C_{jk}$ and 0
otherwise. In the optimisation, Constraint (i) ensures that a task
$j$ is allocated at most to only one coalition (a task does not
need more than one group of responders). Constraint (ii) ensures
that a responder $i$ is assign to only one task (a responder cannot
do more than one task at the same time). This is a standard MILP
that can be efficiently solved by CPLEX.

In order to solve Equation~\ref{eq:cf}, we need to compute the
value of $v(C_{jk})$ for each coalition $C_{jk}$, which is the
long-term value when the responders in $C_{jk}$ are assigned to
task $t_j$. This is challenging because not all tasks can be
completed in one shot and the policy after completing task $t_j$
must be computed as well, which is time-consuming. Alternatively,
we can estimate the value by several simulations. This is much
cheaper because we do not need to compute the complete policy.
According to the central limit theorem, as long as the number of
simulations are sufficient large, the estimated value will converge
to the true coalition value. The main process is outlined in
Algorithm~\ref{alg:tp}.

\begin{algorithm}[t]
  \caption{Task Planning}
  \KwIn{the current state $s$,
  a set of unfinished tasks $T$,
  and a set of free responders $I$.}
  \KwOut{a task assignment for all responders.}
  $\{ C_{jk} \} \gets$ compute all possible coalitions of $I$ for
  $T$ \;
  \ForEach{$C_{jk} \in \{C_{jk}\}$}{
    \tcp{The $N$ trial simulations}
    \For{$i=1$ \KwTo $N$}{
        $(r, s') \gets$ simulate the process with the starting \\\Indp state $s$
        until task $k$ is completed by the responders in $C_{jk}$ \; \Indm
        \If{$s'$ is a terminal state} {
            $v_i(C_{jk}) \gets r$ \;
        } \Else {
            $V(s') \gets$ estimate the value of $s'$ with MCTS \;
            $v_i(C_{jk}) \gets r + \gamma V(s')$ \;
        }
    }
    $v(C_{jk}) \gets \frac{1}{N} \sum_{i=1}^{N} v_i(C_{jk})$ \;
  }
  \Return the task assignment computed by Equation~\ref{eq:cf}
  \label{alg:tp}
\end{algorithm}

In each simulation, we first assign the responders in $C_{jk}$ to
task $t_j$ and run the simulator starting from the current state
$s$. After task $t_j$ is completed, the simulator returns the sum
of the rewards $r$ and the new state $s'$. If all the responders in
$C_{jk}$ are incapable to do other tasks (e.g., receiving too many
radioactive doses), the simulation is terminated. Otherwise, we
estimate the expected value of $s'$ using Monte-Carlo Tree Search
(MCTS), which provides good tradeoff between exploitation and
exploration of the policy space and has been shown to be efficient
for large MDPs~\cite{?}. The basic idea of MCTS is to maintain a
search tree where each node is associated with a state $s$ and each
branch is a task assignment for all responders. After $N$
simulations, the averaged value is returned as an approximation of
the coalition value.

In the task planning level, ``completing a task by a responder'' is
a macro action, assuming that each responder can find the best path
to the task (Section~\ref{sec:pathplanning} gives more detail about
how to compute this). Thus, the main step of implementing MCTS is
to compute an assignment for the free responders (A responder is
free when he or she is capable of doing tasks but not assigned to
any task) at each node of the search tree. This can be computed by
Equation~\ref{eq:cf} using the coalition values estimated by the
UCT heuristic~\cite{?}:
\begin{equation}
  v(C_{jk}) = \overline{v(C_{jk})} + c\sqrt{\frac{2N(s)}{N(s, C_{jk})}}
\end{equation}
where $\overline{v(C_{jk})}$ is the averaged value of coalition
$C_{jk}$ at state $s$ so far, $c$ is a tradeoff constant, $N(s)$ is
the visiting frequency of state $s$, and $N(s, C_{jk})$ is the
frequency that coalition $C_{jk}$ has been selected at state $s$.
Intuitively, if a coalition $C_{jk}$ has bigger averaged value
$\overline{v(C_{jk})}$ or is rarely selected ($N(s, C_{jk})$ is
smaller), it has higher chance to be selected in the next visit of
the tree node.

Once the value of every coalition in $\{ C_{jk} \}$ has been
computed, we solve Equation~\ref{eq:cf} and return the best
assignment of the tasks. One main advantage of our approach is that
it can straightforwardly incorporate the preferences of the
responders. For example, if a responder rejects to do a task, we
simply filter out the coalitions for the task that contain the
responder. By so doing, the responder will not be assigned to the
task. Moreover, if a responder prefers doing tasks with another
responder, we can raise the weights of the coalitions that contain
them in Equation~\ref{eq:cf} (By default, all coalitions have
identical weights of 1.0). Thus, our approach is adaptive to
various preferences of human responders.

\subsection{Path planning}
\label{sec:pathplanning}

\noindent In the path planning, we compute the best path for a
responder given a task assigned to him or her. This path planning
is stochastic as there are uncertainties in the radioactive cloud
and the responders' actions. We model this problem as a
single-agent MDP that can be defined as a tuple, $\mathcal{M}_i =
\langle S_i, A_i, P_i, R_i \rangle$, where:
\begin{itemize}
  \item $S_i = S_r \times S_{p_i}$ is the state space. In this
      level, responder $p_i$ only need to consider the states
      of the radioactive cloud $S_r$ and his or her own states
      $S_{p_i}$ in the MMDP.
  \item $A_i$ is the set of $p_i$'s actions. In this level,
      responder $p_i$ only need to consider his or her moving
      actions.
  \item $P_i = P_r \times P_{p_i}$ is the transition function.
      In this level, responder $p_i$ only need to consider the
      spreading of the radioactive cloud $P_r$ and the changes
      of his or her locations and health levels when moving in
      the filed $P_{p_i}$, which are defined earlier in the
      MMDP.
  \item $R_i$ is the reward function. In this level, responder
      $p_i$ only need to consider the cost of movement and the
      penalty of receiving too many radioactive doses.
\end{itemize}

This is a typical MDP that can be solved by many state-of-the-art
MDP solvers~\cite{?}. Among them, we adopt Real-Time Dynamic
Programming (RTDP)~\cite{?} because it is very efficient for our
problem, a goal-directed MDP with large number of states. Instead
of exploring the whole state space, RTDP only visits the states
that are reachable from the initial state $s^0$ (the start location
of the responder). The main process is outline in
Algorithm~\ref{alg:pp}. If the goal is not reached in a number of
iterations, we assume there is not path between the start location
of the responder and the task location (either there are obstacles
on the path or the responder will be killed by the radioactivity on
the road).

\begin{algorithm}[t]
  \caption{Path Planning}
  \KwIn{the starting state $s^0$ and the goal state $s^g$.}
  \KwOut{a path from the starting location to the goal.}
  $s \gets s^0$ \;
  \Repeat{$s = s^g$}{
    \ForEach{$a\in A_i$}{
        $Q(s, a) \gets R_i(s, a) + \sum_{s'\in S_i} P_i(s'|s, a)
        V(s')$ \;
    }
    $a \gets \arg\max_{a'\in A_i} Q(s, a')$ \;
    $V(s) \gets Q(s, a)$ \;
    $s' \sim P_i(s'|s, a)$ \;
    $s \gets s'$ \;
  }
  \Return{$Q$}
  \label{alg:pp}
\end{algorithm}

There are several techniques we used to speed up the convergency of
RTDP. In our problem, the terrain of the field is static. Thus, we
can initialize the value function $V(s)$ using the cost map
computed offline without considering the radioactive cloud. The
cost map stores the shortest path and the cost value between any
two points in the map. This will help RTDP quickly navigate among
the obstacles (e.g., buildings, water pools, blocked roads) without
getting trapped in dead ends during the search. Another technique
is: when traversing the reachable states (i.e., $s'\in S_i$ in
Algorithm~\ref{alg:pp}), we only consider the responder's current
location and the neighboring points since $P_i(s'|s,a) = 0$ for
other points. This will further speed up the algorithm where the
main bottleneck is the huge state space.


\section{The Atomic Orchid Platform}
Joel and Wenchao
\begin{enumerate}
\item explain the main components  and how agent is integrated
\item explain the instructions given to participants and how it mimics the disaster response problem detailed above.
\end{enumerate}
\subsection{Game scenario}
AtomicOrchid is a location-based mobile game based on the fictitious scenario of radioactive explosions creating expanding and moving radioactive clouds that pose a threat to responders on the ground (the field players), and the targets to be rescued around the game area. Field responders are assigned a specific role (e.g. `medic', `transporter', `soldier', `ambulance') and targets have specific role requirements, so that only certain teams of responders can pick up certain targets. For example, an `injured person' can only be picked up by an `ambulance' and a `medic' together. To pick up targets, the team must be collocated in the immediate proximity of the geofenced target. Furthermore, field responders must not expose themselves to radioactivity from the cloud for too long, else they risk becoming `incapacitated'.

In their mission to rescue all the targets from the radioactive zone, the field responders are supported by (at least one) person in a centrally located HQ room, and the planning agent that sends the next task to the team of field responders [assuming the agent will have been described in detail already].

\subsection{Player interfaces}
Field responders are equipped with a `mobile responder tool' providing sensing and awareness capabilities in three tabs (geiger counter, map, messaging and tasks; see figure XX). One tab shows a reading of radioactivity, player health level (based on exposure), and a GPS-enabled map of the game area to locate fellow responders, the targets to be rescued and the drop off zones for the targets. Another tab provides a broadcast messaging interface to communicate with fellow responders (field responders and HQ). Another tab shows the team and task allocation dynamically provided by the agent. Notifications are used to alert both to new messages and task allocations.

HQ is manned by at least one player who has at their disposal an `HQ dashboard' that provides an overview of the game area, including real-time information of the players' locations (see figure XX). The dashboard provides a broadcast messaging widget, and a player status widget so that the responders' exposure and health levels can be monitored. HQ can further monitor the   current team and task allocations by the agent. Importantly, only HQ has a view of the radioactive cloud, depicted as a heatmap. `Hotter' zones correspond with higher levels of radioactivity.

\subsection{Planning agent}
[Wenchao. Describe how the agent works (not implementation detail, add that in subsection below), i.e., when it is polled, what information is being exchanged, and how the team/task allocation is being constructed from that and sent.]
\subsection{Radiation Cloud Modelling}
The radiation cloud diffusion process is modelled by a nonlinear Markov field stochastic differential equation, which assumes the cloud intensity is Gaussian distributed in log-space.  The cloud is driven by wind forces which vary both spatially and temporally.  Wind forces induce an anisotropic diffusion coefficient into the cloud diffusion process.  The wind velocity is modelled by two a priori independent Gaussian processes (GP), one GP for each Cartesian coordinate axis.  The GP captures both the spatial distribution of the wind velocity and also the dynamic process resulting from shifting wind patterns such as short term gusts and longer term variations.  In our simulation, each spatial wind velocity component is modelled by a squared-exponential GP covariance function, $K$, with fixed input and output scales over time (although any covariance function, stationary or not, can be substituted).

Both the radiation cloud and wind model priors are combined into a single joint model called a {\it latent force model} (LFM)~\cite{alvarez09} and predictions of the radiation cloud intensity are inferred using the extended Kalman filter (EKF).  The EKF provides both the mean and variance of the log-radiation cloud intensity and wind conditions.  Uncertainty arises due to unknown initial conditions of the cloud and wind conditions and it is also induced by the stochastic nature of their processes.  The EKF state $S(t)=(\underline{R}(t) \underline{V}_x(t) \underline{V}_y(t))^T$ represents both Cartesian components of the wind velocity, $V_x(t)$ and $V_y(t)$, and the log-radiation cloud density, $R(t)$, on a regular $N\times M$ grid defined across the environment with grid coordinates $G$.  The temporal component of the wind GP model is assumed Markovian and thus, the wind dynamics are incorporated within the EKF as per the KFGP~\cite{reece10}.  For example, the $N\times M$ x-component of the wind velocity at time-step $t+1$ is $V_x(t+1)=F V_x(t)+\nu_t$, where the process model $F=\rho I$ (where $I$ is the identity matrix) and Gaussian process noise $\nu_t\sim \Bbb{N}(0,(1-\rho^2) K(G,G))$ for correlation, $\rho$, of the wind field between time steps.  When $\rho=1$ the wind velocities are time invariant (although spatially variant).  Values of $\rho<1$ model wind conditions that change over time.

The cloud intensity and wind velocity are measured by {\it monitor agents} equipped with geiger-counters and anemometers.  These agents are directed to take measurements with greatest information gain in the radiation cloud intensity.  The measurements are folded into the EKF and this refines estimates of the radiation cloud across the grid.  Figure~\ref{radiation_screen_shots} shows example cloud simulations for invariant (i.e. $\rho=1.0$) and gusty (i.e. $\rho=0.90$) wind conditions.  Figure~\ref{radiation_screen_shots}(a) shows invariant wind conditions in which case the radiation cloud can be interpolated accurately using sparse sensor measurements and the LFM model.  Alternatively, during gusty conditions the radiation cloud model is more uncertain far from the locations where recent measurements have been taken, as shown in Figure~\ref{radiation_screen_shots}(b).

\begin{figure}[ht] \begin{center}
    \includegraphics[width=0.45\textwidth]{figures/radiation_ss_calm.png}\\
    (a) Slow varying wind conditions\\ \ \\
    \includegraphics[width=0.45\textwidth]{figures/radiation_ss_gust.png}\\
    (b) Gusty wind conditions 
\caption{\label{radiation_screen_shots} Radiation and wind simulation ground truth and EKF estimates obtained using measurements from monitor agents (black dots).  Left most panes are ground truth radiation and wind conditions, the middle panes are corresponding estimates and right most panes are state uncertainties:  (a) Invariant and (b) gusty wind conditions.}
\end{center}
\end{figure}

\subsection{System architecture}
[Wenchao: adapt this to version 2.0] AtomicOrchid is based on the open-sourced geo-fencing game MapAttack\footnote{http://mapattack.org} that has been iteratively developed for a responsive, (relatively) scalable experience.  The location-based game is realized by client-server architecture, relying on real-time data streaming between client and server.

The client-server architecture is depicted in figure XX. Client-side requests for for less dynamic content use HTTP. Frequent events, such as location updates and radiation exposure, are streamed to clients to avoid the overhead of HTTP. In this way, field responders are kept informed in near real-time.

The planning agent agent ... [add implementation detail]

The platform is built using the geoloqi platform, Sinatra for Ruby, and state-of-the-art web technologies such as socket.io, node.js, redis and Synchrony for Sinatra, and the Google Maps API. Open source mobile client apps that are part native, part browser based exist for iPhone and Android; we adapted an Android app to build the mobile responder app.

\section{Pilot Study}
Joel and Wenchao
\begin{enumerate}
\item Explain setup of experiment - area of interest + setup of tasks
\item Explain evaluation = quantitative and qualitative.
\end{enumerate}
\paragraph{Metrics}
\begin{itemize}
\item{Comparisons between with/without agent versions for the below:}
\item{Performance of FR: number of tasks completed, time on task?, number of messages sent, number of teams formed and disbanded, time on team, acknowledgements of tasks}
\item{Messages: classification}
\item{Health}
\item{Distance travelled}
\item{HQ: number of agent monitoring actions (clicks), number of 'supporting'/related messages (e.g., enforcement, contradictions/overriding)}
\item{Agent performance: number of instructions, number of replanning steps, replanning robustness (diversion of task allocation compared to previous step)}
\item{Following instructions ('obedience'): number of instructions followed vs. not followed (incl. number of HQ interventions/overriding agent allocation), instruction handling diagram}
\item
\end{itemize}
\subsection{Conclusions}
\bibliography{citations}
\end{document}
