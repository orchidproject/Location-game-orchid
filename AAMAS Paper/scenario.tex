\section{The Disaster Scenario}\label{sec:scenario}
\noindent We consider a disaster scenario involving a satellite, powered radioactive fuel, that has crashed in a sub-urban area.\footnote{Given the invisibility of radiation, it is possible to create a believable and challenging environment for the responders to solve in our mixed reality game (see Section \ref{sec:atomicorchid}).} Debris is strewn around a large area, damaging buildings and causing accidents and injuring civilians. Moreover, radioactive particles discharged, from the debris are gradually spreading over the area, threatening to contaminate food reserves and people. Hence, emergency services, voluntary organisations, and the military are deployed to help evacuate the casualties and key assets (e.g., food reserves, medication, vehicles) before they are engulfed by the radioactive cloud.  In what follows, we model this scenario formally and then describe the optimisation problem faced by the actors  (i.e., including emergency services, volunteers, medics, and soldiers) in trying to save as many lives and resources as possible.  

%We then propose an algorithm to solve this optimisation problem (in Section \ref{sec:algo}). In Section \ref{sec:atomicorchid}, we demonstrate how this algorithm can be used by a software agent (in our mixed-reality game) in a mixed-initiative process to coordinate first responders. 

\subsection{Formal Model}
\noindent Let $G$ denote a grid overlaid on top of the disaster space, and assume the satellite debris, casualties, assets, and actors are located at various coordinates $(x,y) \in G$ in this grid. The radioactive cloud induces a radioactivity level  $l \in [0,100]$ at every point it covers (100 corresponds to maximum radiation and 0 to no radiation). While the exact radiation levels can be measured by responders on the ground (at a given location) using their geiger counter, we assume that additional information is available  from existing sensors  in the area.\footnote{This assumption is not central to our problem and only serves to inform the decision making of the agent as we see later. It is also possible to obtain similar information about radiation levels by fusing the responders' geiger counter readings but this is beyond the scope of the paper.} However, this information is uncertain due to the poor positioning of the sensors and the variations in wind speed and direction (we show how this uncertainty is captured in the next section). A number of safe zones $G' \subseteq G$ are defined where the responders can drop off assets and casualties (i.e., \emph{targets} to be rescued). Let the set of first responders be denoted as $p_1, \cdots p_i \cdots, p_n \in I$, where $|I| = n$ and the set of  targets to be rescued (i.e., rescue tasks) be denoted as  $t_1,\cdots, t_j, \cdots, t_m\in T$, where $|T| = m$. A rescue task is performed by picking the target up, carrying it to a safe zone, and dropping it off.  As responders perform rescue tasks, they may become tired, get injured, or receive radiation doses that may, at worst, be life threatening. Hence, we assign each responder  a health level $h_i\in [0,100]$ that decreases based on its radiation dose ($h$ is decreased by $0.02 \times h_i $ per second) and assume that its decision to pick up and carry the target allocated to it is liable to some uncertainty (e.g., they may not want to pick a target because it is too far or it is unclear how long it is going to take them to for a sub team to carry it back  to a safe zone).  Moreover, let $Roles$ denote the set of roles that a responders can take (e.g., fire brigade, soldier, or medic)  and this will determine the capabilities he or she has and therefore the tasks he or she can perform. We denote as $r_i \in Roles$ the role of responder $p_i$. In turn, to complete a given task $t_j$,  a set of responders $I' \subseteq I$ with specific roles $R_{t_j} \subseteq R$ is required to pick up . Thus, a task can only be completed by a team of responders $I'$ if $\{Roles(p_i) | p_i \in I'\} = R_{t_j}$.
%\begin{figure}[htbp]
%\includegraphics[width=\columnwidth]{scenario.jpg}\vspace{-5mm}
%
%\label{fig:scenario}
%\caption{The interactions between different actors in the disaster scenario. Lines represent communication links. Planner agent and coordinator sit in the headquarters (HQ). First responders (FRs) can communicate with all actors directly.}\end{figure}
\subsection{Human-Agent Interactions}
\noindent In line with practice in many countries, we assume that the first responders are coordinated from a headquarters headed by a human coordinator $H$. In our case, $H$ is assisted by a planning agent $PA$ that can receive input from and direct the first responders.   Both  $H$ and $PA$  can communicate their  instructions (task allocations) directly to the responders using an instant messaging client (or walkie talkie).  While these instructions may be in natural language for $H$, $PA$ instructs them with simple "Do task X at position Y with team-mate Z" messages. In turn, the responders may not want to do some tasks (for reasons outlined above) and may therefore simply accept or reject the received instruction from $PA$ or $H$. However, $H$ can query the responders' decisions and request for more information about their status (e.g., fatigue or health) and goals (e.g., meeting with team-mate at position X or going for task Y). Instead, if a task is rejected by the responders, $PA$ captures this as a constraint on its task allocation procedure (see more details in Section \ref{sec:algo}) and returns a new plan. Thus on the one hand, richer interactions are possible between $H$ and the first responders than between them and $PA$. On the other hand, $PA$ is powered by state-of-the-art task allocation algorithm that can compute an efficient allocation, possibly better than the one computable by $H$ (particularly when many responders need to be managed). 

It is important to note that our model captures different types of flexible control: (i) agent-based: when $PA$ directly instructs the responders  and they can use multiple iterations of accept/reject to collaboratively converge to a definite  plan (ii) human-based: when $H$ communicates goals to the responders in open-ended interactions that may permit $H$ to gain a better understanding of the context than $PA$ and therefore formulate corrective measures faster. Moreover, in contrast to previous work that suggest \emph{transfer-of-control} regimes \cite{scerri:etal:2005}, our approach does not constrain transfers of control to target specific decision points in the operation of the system. Rather, we our interaction mechanisms are designed (see Section \ref{sec:atomicorchid}) to allow human control at any point (and our results  in Section \ref{sec:evaluation} validate this approach). 

Now, given the above model,  in the next section we describe the MMDP formulation of the coordination problem posed and solve it with an efficient approximate algorithm in Section \ref{sec:algo}.


\subsection{The Optimisation Problem}
\noindent A Multi-agent Markov Decision Process (MMDP) is formally
defined as a tuple $\mathcal{M} = \langle I, S, \{A_i\}, P, R
\rangle$, where:
\begin{itemize}
  \item $I$ is a set of $n$ field responders and each responder
      is associated with a unique identifier number $p_i\in I$.
  \item $S = S_r \times S_{p_1} \times \cdots \times S_{p_n}
      \times S_{t_1} \times \cdots \times S_{t_m}$ is the state
      space. $S_r = \{l_{(x,y)}| (x, y) \in G\}$ is the state
      variable of the radioactive cloud to specify the
      radioactive level $l_{(x,y)}\in[0, 100]$ at every point
      $(x, y)\in G$. $S_{p_i} = \langle h_i, (x_i, y_i), t_j
      \rangle$ is the state variable for each responder $p_i$
      to specify his or her health level $h_i\in[0, 100]$, the
      coordinate $(x_i, y_j)$, and the task $t_j$ carried by
      the responder. $S_{t_j} = \langle st_j, (x_j, y_j)
      \rangle$ is the state variable for task $t_j$ to specify
      its status $st_j$ (picked up, dropped off, or idle) and
      coordinate $(x_j, y_j)$.
  \item $A_i$ is a set of responder $p_i$'s actions. Each
      responder can {\em stay} in the current location $(x_i,
      y_i)$, {\em move} to the 8 neighbouring locations, or
      {\em complete} a task located in $(x_i, y_i)$. A joint
      action $\vec{a}=\langle a_1, \cdots, a_n \rangle$ is a
      set of actions where $a_i\in A_i$, one for each
      responder.
  \item $P = P_r \times P_{p_1} \times P_{p_n} \times P_{t_1}
      \times P_{t_n}$ is the transition function.
      $P_r(s'_r|s_r)$ is the probability for the radioactive
      cloud to spread from state $s_r$ to $s'_r$. It caputers
      the uncertainty of the next radioactive levels of the
      environment due to the noisy sensor reading and the
      variation in wind speed and direction.
      $P_{p_i}(s'_{p_i}|s, a_i)$ is the probability for
      responder $p_i$ to transit to a new state $s'_{p_i}$ when
      executing action $a_i$. For example, when a responder is
      asked to go to a new location, he or she may not be there
      because he or she becomes tired, gets injured, or
      receives radiation doses that are life threatening.
      $P_{t_j}(s'_{t_j}|s, \vec{a})$ is the probability for
      task $t_j$. A task $t_j$ can only be completed by a team
      of responders with required roles locating in the same
      coordinate as $t_j$.
  \item $R$ is the reward function. If a task is completed, the
      team will be rewarded. There will be a penalty for the
      team if any responder gets injured or receives too many
      radioactive doses. Each action of the responders has a
      cost since it will consume energy of the responders.
\end{itemize}
A policy for the MMDP is a mapping from states to joint actions,
$\pi: S \rightarrow \vec{A}$ so that the responders know which
actions to take given the current state of the problem. The quality
of a policy $\pi$ is usually measured by its expected value
$V^\pi$, which can be computed recursively by the Bellman equation:
\begin{equation}
  V^\pi(s) = R(s, \pi(s)) + \gamma\sum_{s'\in S} P(s'|s, \pi(s)) V^\pi(s')
\end{equation}
where $\pi(s)$ is a joint action given $s$ and $\gamma\in(0, 1]$ is
the discounted factor. The goal of solving the MMDP is to find an
optimal policy $\pi^*$ that maximises the expected value with the
initial state $s^0$, $\pi^* = \arg\max_{\pi} V^\pi(s^0)$.

At each decision step, we assume the planning agent can fully
observe the state of the environment $s$ by collecting sensor
reading of the radioactive cloud, GPS data of the responders, etc.
Given a policy $\pi$ of the MMDP, a joint action $\vec{a}=\pi(s)$
can be selected and broadcasted to the responders (as mentioned
earlier). By so doing, each responder can be instructed by the
agent and know how to act in the field.

