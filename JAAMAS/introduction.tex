\section{Introduction}
In the aftermath of major disasters (man-made or natural), first responders  (FRs), such as medics, security personnel and search and rescue teams, are  rapidly dispatched to help save lives and infrastructure. In particular, FRs, with different skills, capabilities and experience may be required for the different tasks that need to be performed.  For example, finding out where the civilians are  trapped requires search and rescue teams and medics, transporting them to safe houses  requires ambulances and security personnel and medics, while  removing dangerous material from the environment requires safety experts and security personnel. While performing such tasks, FRs often operate in a very dynamic and uncertain environment, where, for example, fires  spread, riots start, or the environment floods. Given this, FRs find it difficult to determine the best course of action and which task should be allocated to which team.

To assist in such situations, over the last few years, a number of  algorithms  and mechanisms have been developed to solve the coordination challenges faced by emergency responders  (see Section \ref{sec:relatedwork} for more details). For example, \cite{ramchurn:etal:2010} provide an algorithm to compute the optimal teams of emergency responders to allocate to tasks that require specific types of skills to complete, while \cite{Chapman2009,puyol:etal:2014} distribute such computations in an attempt to reduce the bandwidth required to coordinate. However, none of these approaches consider the inherent uncertainty in the environment or in the first responders' abilities. Crucially, to date, while all of these algorithms have been shown to perform well in simulations, none of them have been \emph{exercised} to guide \emph{real} human responders in real-time rescue missions. Crucially,  studies on the deployment of such intelligent technologies in the real-world reveal that they typically impose a cognitive burden on the first responders \cite{Rachlin1997,Moran2013} and disrupt task performance.  Hence, it is important to develop real-world simulations of disaster response where such technologies can be trialled so that  the interactional issues between humans and agents may be explored. Moreover, only through such trials will it become clear whether these algorithms will cope with real-world uncertainties (e.g., communication breakdowns or changes in weather conditions), be acceptable to humans (i.e., take into account their capabilities and preferences to perform certain tasks), and actually augment, rather than hinder,  human performance (e.g., providing useful guidance and support rather than intrusive ones). 


%Disasters such as the earthquake that hit Haiti in 2010 (resulting in X no. of casualties, X buildings damaged), or the 9/11 terror attack (resulting in X casualties, and several buildings affected), have shown that the way the response effort is organised has a major impact on the recovery process. For example,  in Haiti,  first responders from around the world used web and mobile technologies to map the disaster, allocate resources, and respond to requests for help very efficiently, despite no prior interactions among them \cite{[ushahidi]}.\footnote{For example, the Ushahidi\footnote{\url{http://haiti.ushahidi.com}})  platform was used to share information (e.g., using Twitter and mobile phone texts) between different rescue agencies as well as with the local population.} In contrast, during the 9/11 terror attacks, even though emergency services were all based in Manhattan, they suffered several of their own casualties due to poorly coordinated actions \cite{911report}.  In a recent report, the UNOCHA (the United Nations),  advocated the use of novel web and mobile  technologies in the field to support first responders gather  data before, during, and after disasters in order to efficiently assign tasks and resources to different agencies. 


%Moreover, in order to stimulate research into algorithms for emergency response planning, the  RoboCupRescue simulation platform was developed\footnote{This platform was developed by Japanese researchers in 1995 (after the Hanshin-Awaji earthquake.} to recreate the computational challenges faced by emergency responders during major disasters. In this simulation platform, first responders are represented as software agents of different types (ambulance, fire, and police) that need to coordinate to assign tasks to groups of agents (e.g., digging out civilians from rubble, exitinguishing fires, and unblocking roads), as fires spread and new tasks are discovered. As a result, over the last few years, 
%In turn, in the Human-Computer-Interaction litert
%The coordination of teams of first responders in search and rescue missions is a grand challenge for multi-agent systems research \cite{kitano:2001}. In such settings, responders with different capabilities (e.g., fire-fighting or life support) have to form teams in order to perform rescue tasks (e.g., extinguishing a fire or providing first aid) to minimise  loss of life and costs (e.g., time or money). Thus, responders have to plan their paths to the tasks (as these may be distributed in space) and form specific teams  to complete them. These teams, in turn, may  need to disband and reform in different configurations to complete new tasks, taking into account the status  of the current tasks  (e.g., health of victims or building fire) and the environment (e.g., if a fire or radioactive cloud is spreading). Furthermore, uncertainty in the environment (e.g., wind direction or speed) or in the responders' abilities to complete tasks (e.g., some may be tired or get hurt) means that plans are likely to change continually to reflect the prevailing assessment of the situation. 


Against this background, we develop a novel algorithm for team coordination under uncertainty and evaluate it within a real-world mixed-reality game that embodies the simulation of team coordination in disaster response settings. Specifically, we consider a scenario involving rescue tasks (involving carrying a specific object to a safe zone) distributed in a physical space over which a (virtual) radioactive cloud is spreading.  Tasks need to be completed by pairs of FRs with specific roles (e.g., medic, soldier, fire fighter, or transporter) that need to plan paths from their individual locations to meet at specific points in the disaster space to undertake these tasks. Moreover, they have to do so before the area is completely covered by the cloud (as FRs will die from radiation exposure) which is spreading according to varying wind speed and direction (that may result in highly uncertain radiation level predictions). Our algorithm captures the uncertainty in the scenario (i.e., in terms of environment and player states) and  is able to compute a policy to allocate responders to tasks that minimises task completion time and plans routes for responders to ensure they not exposed to significant radiation. In order to be responsive, our algorithm is designed to return approximate solutions rather than optimal ones (that would take too much time to return solutions in a real-time setting).  The algorithm is then used by a planning agent, working alongside a human commander, to guide FRs  on the ground. Specifically, the planning agent is integrated  into our test platform, AtomicOrchid,\footnote{\url{http://bit.ly/1ebNYty}.} that structures the interaction between FRs (i.e., players on the ground), a human commander, and the planning agent in a mixed-reality location-based game. In particular, the planning agent is designed to take over the burden of computing team plans from the human commander (who takes up a more supervisory role) while being responsive to  requests to change plans from FRs (e.g., in case they are tired or prefer to do other tasks). By so doing, we are able to study, both quantitatively and qualitatively, the performance of a human-agent collective (i.e., a mixed-initiative team where control can shift between humans and agents)  and the interactions between the different actors in the system  \cite{jennings:etal:2014}. In particular, we  advance the state of the art in the following ways:
\begin{enumerate}
\item We develop a Multi-Agent Markov Decision Process (MMDP) to represent the problem of  team coordination  (i.e., path planning and task allocation) under uncertainty \cite{boutilier1996planning} and provide a novel algorithm to compute approximate solutions to the MMDP.  We embed the mechanism to drive a planning  agent in the AtomicOrchid game to evaluate it with users in a real-world setting. 
\item We present a novel mixed-reality game, AtomicOrchid, to evaluate team coordination under uncertainty, focussing particularly on human-agent collaboration. In AtomicOrchid, human players in the field are supported by our planning agent in their mission to coordinate rescue tasks efficiently by communicating with headquarters and each other via a mobile phone application.
%a novel game to evaluate team coordination under uncertainty using the concept of mixed-reality games. 
\item We run field trials of our planning  agent in AtomicOrchid where  instructs field responders through mobile messages in a disaster response scenario in multiple field trials. Our quantitative and qualitative analysis of the results show that providing flexible interactions between human participants and the planning agent improve task performance, particularly when the agent can rapidly respond to human requests for tasks. 
\end{enumerate}
When taken together, our results show, for the first time, how agent-based coordination algorithms for disaster response can be integrated and validated with human teams. Moreover, these results allow us to derive a methodology and guidelines  for systems involving  human-agent collaboration. 

The rest of this paper is structured as follows. Section \ref{sec:scenario} formalises the disaster response problem as a MMDP. Section \ref{sec:algo}  describes the algorithm to solve the path planning and task allocation problems presented by the MMDP. Section \ref{sec:atomicorchid}  details the AtomicOrchid platform. Section \ref{sec:evaluation} presents our pilot study and the  field trial evaluation.  Finally, Section \ref{sec:conclusions} concludes.