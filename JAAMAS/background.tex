\section{Background}
In this section we present related work on multi-agent coordination for disaster response. We particularly focus on two key areas: (i) algorithmic approaches to coordinating emergency responders typically developed by the multi-agent systems community and (ii) decision support systems typically studied by the human factors and human-computer interaction community. By so doing, we also how our work fits at the intersection of these approaches.
\subsection{Coordinating with Decision-Theoretic Planning}

A {\em Markov decision process} (MDP) is a mathematical framework
for sequential decision making under uncertainty. In the presence
of multiple agents, this model has been extended to {\em
multi-agent MDP} (MMDP)~\cite{boutilier1996planning} where the
action chosen at any state consists of individual action components
performed by the agents. Theoretically, any algorithm such as {\em
linear programming}, {\em value iteration}, or {\em policy
iteration} that can solve MDPs can also be used to solve MMDPs.
However, it will be very inefficient because the action space grows
exponentially with the number of agents. To exploit the domain
structure, {\em factored MDP} (FMDP)~\cite{boutilier2000stochastic}
was introduced in which the state space is described by a set of
variables and the transition model is defined by a {\em dynamic
Bayesian network} (DBN). When the agents can only observe partial
information of the state, this problem can be modeled by {\em
multi-agent partially observable MDPs}
(MPOMDP)~\cite{pynadath2002communicative}. Similar to MMDP, MPOMDP
can be treated as an extension of single-agent POMDP to multi-agent
domains. This analogy is useful because MPOMDPs can be solved as
belief-state MMDPs where a belief state is a probability
distribution over the states. All the above models assume that
there is a centralised unit that will select a joint action for the
team and distribute each action component to the corresponding
agent. {\em Decentralised POMDP}
(DEC-POMDP)~\cite{bernstein2002complexity} is a more general model
where the agents are controlled in a decentralised manner. In other
words, there is no centralised unit for distributing the actions
and each agent must choose its own action based on the local
observation.

In this article, we restrict ourself to model our problem as the
MMDP because other models do not fit the characteristic of our
domain or are too difficult to be solved with the size of our
problem. Specifically, in our domain, there is a headquarter that
will collect all the information and distribute the commands to
each filed responder. Therefore, it is not necessary to assume that
the information is only partial (as in MPOMDPs) or the decision
must be made locally by the responders (as in DEC-POMDPs).
Furthermore, those models are much harder than MMDPs and the
existing algorithms can only solve very small problems. We do not
use the FMDP because most of the algorithms for solving this model
require that the value function can be factored additively into a
set of localized value
functions~\cite{koller2000policy,guestrin2001multiagent,guestrin2003efficient}
and our problem does not have such structures. For example, in our
domain, several tasks may depend on the same responder. If she is
delayed in one task, this may affect the completion of the other
tasks. In other words, the completion of one task may depend on the
completion of the other tasks so the value function can not be
factored on the basis of the local task states. Our settings are
also different from the one in~\cite{Chapman2009} where they assume
that the responders are self-interested and need to negotiate with
each other on the task that they want to perform next.

As aforementioned, any algorithm that can solve large MDPs can be
used to solve MMDPs. However, most of the existing approaches are
offline algorithms (See the most recent
survey~\cite{kolobov2012planning} for more detail). The main
disadvantage of offline algorithms is that they must compute a
complete action mapping for all possible states in the policy. This
is intractable for problems with huge state space as in our domain.
In contrast to offline approaches, online algorithms interleave
planning with execution and only need to compute the best action
for the current state instead of the entire state space.
Specifically, we adopt the basic framework of {\em Monte-Carlo tree
search} (MCTS)~\cite{kocsis2006bandit}, which is currently the
leading online planning algorithm for large MDPs, and divide our
online algorithm into two levels: task planning and path planning.
It is worth pointing out that our method is different from the
hierarchical planning for MMDPs~\cite{musliner2006coordinated}
because it requires the task hierarchy to be part of the model and
our problem does not have such task hierarchy for the responders.
Indeed, our problem is more closely related to the {\em coalition
formation with spatial and temporal constraints} (CFST) problem
where agents form coalitions to complete tasks, each with different
demands. However, existing work on CFST often assumes that there is
no uncertainty on the agents' actions and the
environment~\cite{ramchurn:etal:2010}.

\todo{Agents for Disaster Response}
Kitano et al \cite{kitano:XX} were the first to propose disaster response as a key application area for multi-agent systems. Since then, a number of algorithms and simulation platforms have been developed to solve the computational challenges involved. For example, a number algorithms such as have been developed to efficiently allocate emergency responders to rescue tasks (e.g., to rescue civilians, extinguish fires, or unblock roads) for (i) decentralised coordination: where emergency responders need to chose their actions based on local knowledge \cite{chapman:etal,puyol:etal}, (ii) centralised coordination: where a command centre is able to choose actions for all the members of the team given complete knowledge of the system \cite{koes:etal,scerri:etal,boloni} , and (iii) coalition formation: to choose the optimal teams to allocate to specific tasks \cite{ramchurn:etal}.Many of these algorithms are actually evaluated in simulation using the the RoboCupRescue disaster simulation platform \cite{rescue}. In this platform, emergency response tasks are modelled computationally (e.g., functions describing speed and efficiency of agents at completing tasks) and the emergency responders are modelled as agents that automatically implement the outputs of a given task allocation algorithm \cite{kleiner:etal,ramchurn:etal}. While such evaluations are useful to determine extreme scenarios (e.g., best case when all agents implement all tasks perfectly or worst case when they do not), they are prone to misrepresentations of human decision making. 

\begin{enumerate}
\item Gopal's background on coordination for disaster response, RoboCup Rescue, and Scerri's work.
\item Feng's background on MDP approaches to coordination (incl CFST and Archie's paper)
\item Joel/Wenchao to add background on HCI studies on coordination in disasters + mixed reality 
\end{enumerate}}

\subsection{ Team Coordination and Disaster Response}
Team coordination can be defined as “the act of managing interdependencies between activities performed to achieve a goal” Malone (1990: 361). In disaster response, team coordination is essential in order that groups of people can carry out interdependent activities together in a timely and satisfactory manner (cf. Bradshaw et al., 2011). Disaster response experts report that “failures in team coordination are the most significant factor in critical emergency response” (Toups et al., 2011: 2) that can cost human lives. Shared understanding, situation awareness, and alignment of cooperative action through on-going communication are key requirements to enable successful coordination. Literatures (Monares, 2011, Padilha, 2010, Convertino et al. 2011) have proposed a set of tools and system architectures that support common ground and awareness in emergency management. \\

(Chen et al., 2005) highlighted that One important characteristic of large-scale disaster is the presence of multiple spatially distributed incidents . To deal with multiple incidents, the disaster response team has to coordinate spatially distributed resources and personnel to carry out operations (e.g. search, rescue and evacuation). Therefore, it is necessary to optimize coordination of teams by allocating tasks to teams in time and space efficiently and sufficiently (Nourjou et al 2011). \\

\subsection{ Issues of Human Agent Collaboration }
Many multi-agent agent coordination algorithms have potential to be applied to support task assignment of responder teams. However, before we use those algorithms to build agent planning support system, there is a need to understand how human and agent support system can effectively collaborate.  Human factors researchers have conducted controlled experiments to identify key aspects of human agent collaboration [ Bradshaw 2011, Cooke2007 ,Sukthankar2009, Wagner2004] and evaluate strategies of agent support for teams [Lenox1998,Lenox2000,(Nourjou et al 2011) ]. Prior research has recognised that interaction design is vital for the performance of socio-technical human-agent systems [Murthy1997], particularly where an agent directly instructs humans [S. Moran2013]. With inappropriate interaction design, agent-based planning support may function inefficiently, or at worst, hinder the performance of human teams. Although there is much literature in planning support, task assignment, and human-agent collaboration, Yet, real world studies of how human teams handle agent support are rare. \\

Moverover, CSCW literatures[Bowers1994] have pointed out ill-designed work-flow management/automation system can lead to undesirable results, not only fail to improve work efficiency but also hinders human performance. Bowers et al. found that extreme difficulties might be encountered when introducing new technology support for human teams. New technologies might not support, but  disrupt smooth workflow if they are designed in an organisationally unacceptable way. [Abbott1994] We believe the same is true for intelligent planning support. Before we can build intelligent systems that support human team coordination, field trials are needed to understand the potential impact of technology support for team coordination. \\

\subsection{ Disaster Simulation and Games }

Computational simulations, particularly agent-based simulations of disasters, are the predominant approach in the computing literature to predict the consequences of cer- tain courses of action (Hawe et al., 2012), to model first responder information flow (Robinson and Brown, 2005), or to model the logistic distribution of emergency relief supplies (Lee et al., 2007).\\

The limitations of the veracity of computational simulations are manifold. For ex- ample, Simonovic highlights that simulations may rely on unrealistic geographical topography, and most importantly, may not account for “human psychosocial charac- teristics and individual movement, and (...) learning ability” (Simonovic, 2009: 89). The impact of emotional and physical responses likely in a disaster, such as stress, fear, exertion or panic (Drury et al., 2009) remains understudied in approaches that rely purely on computational simulation.\\


In this study, we adopt a mixed-reality game approach to put people under realistic cognitive and physical stress. Mixed-reality games are recreational experiences that make use of pervasive technologies such as smart phones, wireless technologies and sensors with the aim of blending game events into a real world environment [6]. Arguably, they have become an established vehicle to explore socio-technical issues in complex real world settings [5]. The major advantage of mixed-reality games is the fact that they are situated in the real world, which arguably leads to increased efficacy of the behavioural observations when compared to computational simulations.\\


\todo{
The following are a few suggestions for improving the work.
-- the paper should cite the work by Humphereys and Adams that discussed assignment of responders in emergency situations. Their work is particularly relevant because of the way they balanced competing task demands.
}

\todo{Please add more references to cover beyond what we do to some extent}